
%\documentclass[a4paper,12pt]{amsart}
%\usepackage[a4paper, margin=1in]{geometry}
%\setlength{\parindent}{0cm}
%\setlength{\parskip}{1\baselineskip}
%
%\setcounter{secnumdepth}{4}
%\usepackage{amsthm}
%\usepackage{aliascnt}
%\usepackage{amsmath,amssymb,amsfonts}
%\usepackage{hyperref}
%\usepackage[shortlabels]{enumitem}
%
%\usepackage{tikz-cd}
%
%\newcommand{\basetheorem}[3]{
%	\newtheorem{#1}{#2}[#3]
%	\newtheorem*{#1*}{#2}
%	\expandafter\def\csname #1autorefname\endcsname{#2}
%}
%\newcommand{\maketheorem}[3]{
%	\newaliascnt{#1}{#3}
%	\newtheorem{#1}[#1]{#2}
%	\aliascntresetthe{#1}
%	\expandafter\def\csname #1autorefname\endcsname{#2}
%	\newtheorem{#1*}{#2}
%}
%
%%\theoremstyle{plain}
%\basetheorem{theorem}{Theorem}{section}
%\maketheorem{definition}{Definition}{theorem}
%\maketheorem{lemma}{Lemma}{theorem}
%\maketheorem{example}{Example}{theorem}
%\maketheorem{corollary}{Corollary}{theorem}
%\maketheorem{conjecture}{Conjecture}{theorem}
%\maketheorem{proposition}{Proposition}{theorem}
%\maketheorem{question}{Question}{theorem}
%\maketheorem{remark}{Remark}{theorem}
%
%
%%\newtheorem{theorem}{Theorem}[section]
%%\newtheorem{definition}{Definition}{theorem}
%%\newtheorem{lemma}{Lemma}{theorem}
%%\newtheorem{example}{Example}{theorem}
%%\newtheorem{corollary}{Corollary}{theorem}
%%\newtheorem{conjecture}{Corollary}{theorem}
%%\newtheorem{proposition}{Proposition}{theorem}
%%\newtheorem{question}{Question}{theorem}
%%\newtheorem{remark}{Remark}{theorem}
%
%
%\makeatletter
%\newtheorem*{rep@theorem}{\rep@title}
%\newcommand{\newreptheorem}[2]{%
%	\newenvironment{rep#1}[1]{%
%		\def\rep@title{#2 \ref{##1}}%
%		\begin{rep@theorem}}%
%		{\end{rep@theorem}}}
%\makeatother
%
%\makeatletter
%\@namedef{subjclassname@2020}{%
%	\textup{2020} Mathematics Subject Classification}
%\makeatother
%
%\newreptheorem{theorem}{Theorem}
%
%\newcommand{\A}{\mathcal{A}}
%\newcommand{\B}{\mathcal{B}}
%\newcommand{\C}{\mathcal{C}}
%\newcommand{\E}{\mathcal{E}}
%\newcommand{\F}{\mathcal{F}}
%\newcommand{\PP}{\mathcal{P}}
%\newcommand{\HH}{\mathcal{H}}
%\newcommand{\SB}{\mathbf{SB}}
%\newcommand{\BB}{\mathbf{B}}
%\newcommand{\BS}{\mathbf{B}_{+}}
%\newcommand{\disc}{\text{Discrepancy}}
%
%
%\newcommand{\ta}[1]{\mathcal{A}^{\leq #1}}
%\newcommand{\at}[1]{\mathcal{A}^{\geq #1}}
%\newcommand{\tb}[1]{\mathcal{B}^{\leq #1}}
%\newcommand{\bt}[1]{\mathcal{B}^{\geq #1}}
%\newcommand{\tc}[1]{\mathcal{C}^{\leq #1}}
%\newcommand{\ct}[1]{\mathcal{C}^{\geq #1}}
%\newcommand{\Ht}[1]{H^{i}_{t}}
%\newcommand{\orth}{^{\perp}}
%\newcommand{\Hom}{\text{Hom}}
%\newcommand{\Fe}{F^{e}_{*}}
%\newcommand{\Fn}[1]{F^{#1}_{*}}
%\newcommand{\trip}{(R,\Delta, \alpha_{\bullet})}
%\newcommand{\ai}{\alpha_{\bullet}}
%\newcommand{\im}{\text{Im}}
%\newcommand{\ox}[1][X]{\mathcal{O}_{#1}}
%\newcommand{\me}{M^{e}_{\Delta,a^{t}}}
%\newcommand{\psim}{\sim_{\mathbb{Z}_{(p)}}}
%\newcommand{\zp}{\mathbb{Z}_{(p)}}
%\newcommand{\Xde}[1]{\mathcal{X}_{\delta,\epsilon,#1}}
%\newcommand{\Pde}[1]{\mathcal{P}_{\delta,\epsilon,#1}}
%
%\usepackage{xcolor}
%\newcommand\myworries[1]{\textcolor{red}{#1}}
%
%\newcommand{\coker}{\text{coker }}
%\title{Mori Fibrations in Mixed Characteristic}
%
%\author{Liam Stigant}
%
%\address{Department of Mathematics, Imperial College London, 180 Queen's Gate, 
%	London SW7 2AZ, UK} 
%\email{l.stigant18@imperial.ac.uk}
%
%\subjclass[2020]{14J30, 14E30, 14G45, 14B05}
%\keywords{Sarkisov Program, Minimal Model Program, Mixed Characteristic, Mori Fibre Spaces}
%\begin{document}

	\chapter{Finiteness of Minimal Models}
	\section{Introduction}
	
	Recent work in \cite{bhatt2020globally+} establishes the bulk of the Minimal Model Program for KLT threefold pairs over suitable bases in base in mixed characteristic. In particular it is shown that one can always run an MMP with scaling. When the pair is psuedo-effective more is known, it is shown that in fact every MMP terminates without the need for scaling. A small adaptation of the arguments of \cite{kawamata2008flops} ensures these models are connected by flops. This paper focuses on the outstanding questions for pairs which are not pseudo-effective over mixed characteristic bases. The main restrictions are that the residue fields of $R$ should have characteristic greater than $5$. A full characterisation of suitable base rings is given in \autoref{setup}.
	
	
	
	First it is shown that in fact the threefold MMP always terminates, extending the termination result of \cite{bhatt2020globally+} to pairs which are not pseudo-effective.
	
	
	\begin{theorem}[\autoref{termination}]
		Let $f:(X,\Delta) \to T$ be a threefold klt pair over $R$ then any $K_{X}+\Delta$ MMP terminates.
	\end{theorem}

	An MMP for a pair which is not pseudo-effective will always terminate with a Mori fibre space. Unlike minimal models these are not connected by flops or even isomorphic in codimension $2$. They can be very varied even in dimension $2$. Nonetheless they are conjecturally related by a sequence of elementary transformations called Sarkisov Links. This claim is known as the Sarkisov program. It is shown that any two threefold Mori fibres spaces which are the output of the same MMP are related by Sarkisov links.
	
	\begin{theorem}[\autoref{sarkisov}]
		Let $g_{1}:Y_{1} \to Z_{1}$ and $g_{2}:Y_{2} \to Z_{2}$ be two Sarkisov related Mori fibre spaces over $R$ of dimension $3$. Then they are connected by Sarkisov links.
	\end{theorem}
	
	The proof of this second theorem follows closely the work of \cite{hacon2009sarkisov}. The main technical work comes in proving a suitable version of finiteness of minimal models, which is due to \cite{birkar2010existence} in characteristic $0$. 
	
		\begin{theorem}[\autoref{rltfiniteness}]
		Let $X$ be a threefold over $R$. Let $A$ be a big and nef $\mathbb{Q}$-Cartier divisor and $C$ be a rational polytope inside $\mathcal{L}_{A}(V)$. Suppose there is a boundary $A+B \in \mathcal{L}_{A}(V)$ such that $(X,A+B)$ is a klt pair. Then the following hold:
		
		\begin{enumerate}
			\item There are finitely many birational contractions $\phi_{i}:X \dashrightarrow Y_{i}$ such that 
			\[\mathcal{E}(C) = \bigcup \mathcal{W}_{i}=\mathcal{W}_{\phi_{i}}(C)\]
			where each $\mathcal{W}_{i}$ is a rational polytope. Moreover if $\phi:X \to Y$ is a wlc model for any choice of $\Delta \in \mathcal{E}(C)$ then $\phi=\phi_{i}$ for some $i$, up to composition with an isomorphism.
			
			\item There are finitely many rational maps $\psi_{j}:X \dashrightarrow Z_{j}$ which partition $\mathcal{E}(C)$ into subsets $\mathcal{A}_{\psi_{j}}(C)=\mathcal{A}_{i}$.
			\item  For each $W_{i}$ there is a $j$ such that we can find a morphism $f_{i,j}: Y_{i} \to Z_{j}$ and $W_{i} \subseteq \overline{A_{j}}$.
			\item  $\mathcal{E}(C)$ is a rational polytope and $A_{j}$ is a union of the interiors of finitely many rational polytopes.
		\end{enumerate}
	\end{theorem}
	
	In fact these results hold for a slightly more general class of singularities - rlt pairs, which are essentially pairs which are replaceable by linearly equivalent klt pairs locally over the base. This generalisation is necessary due to the lack of appropriate Bertini type theorems over a general ring. Even if one starts with Mori Fibre Spaces coming from a klt MMP, the Sarkisov links may involve rlt pairs. A full definition of rlt is given in \autoref{rlt-section} and a description of Sarkisov links in \autoref{Sarkisov-section}.
	
	\textbf{Acknowledgments}
	Thanks to Federico Bongiorno and Paolo Cascini for their support and for many useful discussions. Thanks also to the EPSRC for my funding.	
	

	The remainder of the chapter will show that any klt MMP terminates for mixed characteristic threefolds. In particular full termination holds for pairs which are not pseudo-effective.We will heavily rely on the observation, due to \cite{katsura1985elliptic}, that $-1$ curves on the special fibre of a smooth family over a DVR can be contracted after possibly taking a finite extension of the base. We will also regularly use the following idea: If $\pi:X \to Y$ is some extremal contraction over $R$, and there is some base change $R' \to R$ such that $X_{R'}$ admits a divisorial contraction over $Y_{R'}$ then the original contraction must also have been divisorial.

	We will also need the following construction, essentially due to \cite{mumford1961topology}.
	
	\begin{lemma}
		
		Let $\pi:X \to Y$ be a projective morphism from a regular scheme to a normal scheme, both of dimension $2$, with geometrically connected fibres. Let $E_{1},...,E_{n}$ be the exceptional curves. Choose a divisor $D$ on $Y$ and write $D'$ for the strict transform of $D$. Then there are unique $m_{i} \geq 0$ with $D'+\sum m_{i}E_{i} \equiv_{Y} 0$. If $D$ is $\mathbb{Q}$-Cartier then we have $\pi^{*}D= D'+\sum m_{i}E_{i}$.
		
	\end{lemma}
	
	\begin{proof}
		
		By \cite[Theorem 10.1]{kk-singbook}, the intersection form $[E_{i}.E_{j}]$ is negative definite. Hence there is a unique choice of $m_{i}$ with $D'+\sum m_{i}E_{i} \equiv_{Y} 0$. It remains to show that $m_{i} \geq 0$. Now suppose for contradiction that $m_{k} < 0$ for some $k$. Then we may suppose that $m_{k}/r_{k}$ is minimal, otherwise if $m_{j}/r_{j}$ is minimal we just replace $k$ with $j$ as we must still have $m_{j} < 0$.
		
		By \cite[Lemma 10.2]{kk-singbook} there is $E= \sum r_{i}E_{i}$ effective on $X$ with $-E$ ample over $Y$. Then $E.E_{i} < 0$ for each $i$ ensures that $r_{i} > 0$ for all $i$.
				
		We must have, for every $j$, that $D'.E_{j}\geq 0 $ as it does not contain any $E_{j}$ and thus as $D' \equiv_{Y} - \sum m_{i}E_{i}$ we have
		
		\[0\geq (\sum_{i} m_{i} E_{i}).E_{j} = \sum_{I} \frac{m_{i}}{r_{i}}(r_{i}E_{i}.E_{j}) \geq \frac{m_{k}}{r_{k}} \sum_{i} (r_{i}E_{i}.E_{j}) > 0\]
		
		This is a contradiction and hence in fact $m_{i} \geq 0$ for each $i$. That this agrees with the pullback when $D$ is $\mathbb{Q}$-Cartier is immediate from uniqueness.
		
	\end{proof}
	
	\begin{lemma}\label{num-pull}
		
		Let $X$ be an $\mathbb{Q}$-factorial scheme together with a projective morphism $f:X \to Y$  with geometrically connected fibres to an excellent normal scheme of dimension $2$. Suppose $V$ is a closed subscheme of $X$ with $f(V)$ contained in a divisor $D$. Then there is a divisor $D'$ on $X$ lying over $D$ and containing $V$.
		
		\end{lemma}
	
	\begin{proof}
		
		Let $\pi \colon Y' \to Y$ be a resolution of $Y$ and $X'$ be the normalisation of the dominant component of the fibre product $X\times_{Y} Y'$. From above we have $F$ on $Y$ lying over $D$ with $F \equiv_{Y} 0$. We have induced maps $g\colon X' \to Y'$ and $\phi\colon  X' \to X$. Now $g^{*}F$ is numerically trivial over $Y$, and hence over $X$. Thus as $X$ is $\mathbb{Q}$-factorial there is $D'$ with $\phi^{*}D'=F$. It is clear from the construction that $f_{*}D'=\pi_{*}F=D$. Suppose that $C$ is a curve lying over $D$, then we must have $D'.C =0$. If $C$ is not contained in $D'$ then since $f$ has connected fibres we may suppose that $D'$ meets $C$, up to replacing $C$ with another curve in the same fibre, but then $D'.C > 0$, a contradiction. Hence $D'$ contains every curve, and hence every fibre, over $D$. In particular it contains $V$.
		
		
	\end{proof}
	
	\begin{lemma}\cite[Lemma 9.4]{katsura1985elliptic} \label{KU}
		Let $R$ be a DVR with algebraically closed residue field $k$. Take $X$ a smooth threefold over $R$ and suppose $C$ is a $-1$ curve on $X_{k}$. Then there is a DVR $R' \supseteq R$ with residue field $k$ such that $X'=X\times R$ admits a proper surjective morphism of algebraic spaces $\pi:X' \to Y$  contracting $C$. Moreover $\pi$ contracts a $-1$ curve $C'$ on the generic fibre of $X' \to X$ which specialises to $C$.
	\end{lemma}
	
	The observation that $R'$ can be chosen to have the same residue field as $R$ is due to \cite[Lemma 3.6]{egbert2016log}.
	
	\begin{theorem}\label{curves}
		Let $R$ be a DVR and $X \to R$ be a smooth threefold. Suppose $C$ is an extremal $K_{X}$ negative curve and that the contraction of $C$, $X\to Y$ is birational, then $C$ lifts to a family of curves $\tilde{C}$ on $X$. 
	\end{theorem}
	
	\begin{proof}
		
		Let $\pi:X \to Y$ be the contraction of $C$. 
		
		If $\kappa$ is not closed then we can take an inclusion $R \hookrightarrow R'$ so that $R'$ is a DVR with residue field $\bar{\kappa}$. Then $X'=R'\times X$ is still smooth and $X' \to Y'$ is divisorial if and only if $X \to Y$ is. Then $X'_{k} \to Y'_{k}$ is birational and $k_{X}$ negative, so there is some $-1$ curve, $C'$ contracted by $X'_{k} \to Y'_{k}$. 
		
		Now by \autoref{KU} we can find an extension $R \subseteq R'$ with residue field $\kappa$ such $X''=X'\times R'$ admits a divisorial contraction which corresponds to a contraction of $C'$, viewed here as a curve on $X''$. By construction, however $\pi':X'' \to Y''=Y'\times R'$ must factor as $X''\xrightarrow{\phi} Z \to Y''$, and hence $X' \to Y'$ is necessarily divisorial.
		
		Let $E$ be a prime divisor contracted by $X' \to Y'$, then the image $\tilde{C}$ on $X$ is a divisor contracted by $X \to Y$. In particular $\tilde{C} \cap X_{k}$ contains $C$. On the other hand $E\cap X'_{k}$ is irreducible, hence so too is $\tilde{C} \cap X_{k}$. Hence $\tilde{C}$ specialises to $C$ as claimed.

	\end{proof}

	\begin{lemma}\label{lifting}
		
		Let $X \to T$ be a klt threefold over a $R$. Suppose that $X$ has a terminalisation which is smooth over $T$, then any birational $K_{X}$ negative extremal contraction is divisorial.
		
	\end{lemma}
	
	\begin{proof}
		
		Since $X$ is normal we may freely replace $T$ with its normalisation.
		
		Let $h:X \to Z$ be a birational contraction of a $K_{X}$ negative extremal ray. Let $C$ be a curve contracted by $h$. Let $\pi:Y \to X$ be the smooth terminalisation. Then $K_{Y}+\Delta=\pi^{*}K_{X}$ for some exceptional $\Delta \geq 0$. Let $C'$ be a curve dominating $C$. Then $(K_{Y}+\Delta).C' <0$ and thus $K_{Y}.C' < 0$ since $C'$ cannot be contained in the support of $\Delta$ as $Y$ has no vertical exceptional divisors, because it is smooth over $R$.
		
		By assumption $C'$ is contained in a smooth fibre $Y_{p}$ of $Y \to T$. It cannot be the case that $\dim Y_{p}=1$ else $C=Y_{p}$ and $h$ is not birational. Thus we may assume that $\dim Y_{p}=2$ and consequently that $\dim R=1$. Then $Y_{p} \to Z_{p}$ is birational and $K_{Y_{p}}$ is not nef over $Z_{p}$, since $K_{Y_{p}}.C' < 0$. Thus we have an extremal $K_{Y_{p}}$ negative contraction $g:Y_{p} \to Z'$. This is necessarily birational, and hence contracts a curve $\Gamma$.
		
		We have the following diagram.
		
		\[\begin{tikzcd}
		& Y_{p} \arrow[ld, "g"] \arrow[r]  \arrow[d, "\pi_{p}"]& Y \arrow[d, "\pi"] \\
		Z' \arrow[rd]         & X_{p} \arrow[r] \arrow[d]         & X \arrow[d, "h"]   \\
		& Z_{p} \arrow[r]                & Z                 
		\end{tikzcd}\]
		
		
		By \autoref{curves}, therefore, $\Gamma$ lifts to a family of curves $\tilde{\Gamma}$ on $Y$. We can push $\Gamma$ forward to $\Gamma'$ on $X$. Then $\Gamma'$ is contracted by $h$ as $g$ contracts $\Gamma$. However $h$ must therefore contract $\pi_{*}\tilde{\Gamma}$, so it is divisorial as claimed. 		
	\end{proof}

	
	\begin{theorem}
		Let $(X,\Delta) \to T$ be a threefold klt $R$-pair such that $X$ is supported over infinitely many closed points of $T$. Then there is some non-empty open set $U \subseteq T$ such that if $h:X \to X'$ is an extremal birational contraction which contracts a curve $C\subseteq X_{U}=X\times U$ then $h$ is a divisorial contraction.
	\end{theorem}
	
	\begin{proof}
		
		Let $X' \to X$ be a terminalisation of $X$. Then since $X'$ is terminal, it has isolated singularities and thus the generic fibre over $T$, $X'_{K}$, is regular. Since $K$ has characteristic $0$, $X'_{K}$ is necessarily smooth. Hence as smoothness is an open condition, and since $f(X)$ contains infinitely many points, there is a non-trivial open set $U \subseteq T$ such that $X'_{U}$ is smooth over $U$. Shrinking $U$ we may suppose that every component of $\Delta$ either has irreducible fibres over $U$ or does not meet $X_{U}$. 
		
		Take $h:X \to Z$ be an extremal $(K_{X}+\Delta)$ negative birational contraction over $T$ and suppose it contracts $C \subseteq X_{U}$. 
		
		If $C \subseteq \Delta_{i}$ for some component $\Delta_{i}$ of $\Delta$ then since $\Delta_{i}$ has irreducible fibres we must contract the whole divisor, in particular $h$ is divisorial.
		
		Suppose instead $C \nsubseteq \text{Supp}(\Delta)$ then we must have $\Delta.C \geq 0$ and hence $K_{X}.C < 0$. So $h$ is a $K_{X}$ negative extremal contraction. However by construction $X_{U}$ admits a smooth terminalisation $X'_{U}$ and so by \autoref{lifting} we must then have that $h$ is divisorial.
	\end{proof}
	
	\begin{corollary}\label{termination}
		Let $f:(X,\Delta) \to T$ be a $\mathbb{Q}$-factorial threefold klt pair over $R$ then any $K_{X}+\Delta$ MMP terminates.
	\end{corollary}
	
	\begin{proof}
		
		It is enough to show there is no infinite sequence of flips. If $\dim T=3$ then every pair is effective, so this follows immediately from \cite[Theorem F]{bhatt2020globally+}, so suppose that $\dim T \leq 2$.
		
		There is always some open divisor on $T$ such that all the flips take place over $D$. If $X$ is supported over only finitely many points of $R$ then any divisor $D$ which contains them all will suffices. Otherwise we apply the above lemma to see all the flips take place outside some open set $U \subseteq R$. Shrinking $U$ if needed we may suppose that $D=R \setminus U$ is a divisor as claimed.

		If $T$ is $\mathbb{Q}$-factorial then $(X,\Delta'=\Delta+tf^{*}D)$ is klt for small $t$ and a $K_{X}+\Delta$ MMP is also a $K_{X}+\Delta'$ MMP. Since all the flips are contained in the support of $\Delta'$ the sequence must terminate. We may take a normalisation of $T$ if necessary, this shows terminalisation when $\dim T=1$,
		
		If instead $\dim T =2$ , then we may replace $f^{*}D$ in the argument above with $D'$ as found in \autoref{num-pull} and the result follows.
		
	\end{proof}
	
	
	
	\section{Relatively Log Terminal Pairs} \label{rlt-section}


	Here we introduce relatively log terminal pairs, which are essentially pairs which are replaceable by a klt pair locally over the base, and verify that the main results of the MMP extends to this setting. A suitable Bertini type theorem is also established.
	
	\begin{definition}
		We say an $R$-pair $(X,\Delta)/T$ is relatively log terminal (rlt) (resp. relatively log canonical (rlc)) there is a finite open cover $U_{i}$ of $T$ such that on each $X_{i}=U_{i} \times X$ we have $(K_{X}+\Delta)|_{U_{i}} \sim_{\mathbb{R}} K_{X_{i}}+\Delta_{i}$ where $(X_{i},\Delta_{i})$ is a klt (resp. rlc) pair. In this case we say that $(X,\Delta)$ is witnessed by $(X_{i},\Delta_{i})$. We also sometimes say $\Delta$ is witnessed over $U_{i}$. 
		
		If $S \subseteq WDiv(X)$ then we say $(X,\Delta)$ is rlt (resp. rlc) with witnesses in $S$ if $\Delta_{i} \in S|_{U_{i}}$ for each $i$ for some choice of witnesses.
	\end{definition}
	\begin{remark}
		$T$ is always quasi-compact so this is equivalent to asking for $K_{X}+\Delta \sim K_{X_{p}}+\Delta_{p}$ with $(X_{p},\Delta_{p})$ klt for each $p \in T$ where $X_{p}=X \times T_{p}$ for $T_{p}$ the localisation at $p$.
	\end{remark}
	
	Being rlt can be quite a sensitive condition. In particular it's not true that if $B \leq B'$ and $(X,B')$ is rlt that $(X,B)$ must be rlt. For example, for any choice of $B$ and sufficiently ample $H$, on $X$ klt and $\mathbb{Q}$-factorial, we have that $(X,B+H)$ is rlt, though $B$ might not be.
	
	It fits well in the context of polytopes however as if $B_{i}$ are rlt then so is $\sum_{1}^{n} \lambda_{i}B_{i}$ for any choices of $\lambda_{i} \geq 0$ with $\sum \lambda_{i} \leq 1$.
	
	The psuedoeffective cone is the closure of the big cone, and $D$ is big if and only if its pullback to the generic fibre of $X \to T$ is. Hence if $U_{i}$ is any open cover of $T$, then $D$ is psuedoeffective if and only if $D|_{U_{i}}$ is. In particular an rlc pair is psuedoeffective (resp. big) if and only if its witnesses are.
	
	\begin{definition}
		Let $\phi:X \dashrightarrow Y$ be a birational contraction. Take a divisor $D$ and write $D'=\phi_{*}D$. 
		
		We say it is $D$-non-positive (resp. $D$-negative) if there is a common resolution $p:W \to X$, $q:W \to Y$ where 
		
		\[p^{*}D=q^{*}D'+E\]
		and $E \geq 0$ is $q$ exceptional (resp. $E \geq 0$ is $q$ exceptional and contains the strict transform of every $\phi$ exceptional divisor in its support). 
		
		If $(X,\Delta)$ is a psuedoeffective lc pair then $\phi$ is a weak log canonical (wlc) model if $\phi$ is $K_{X}+\Delta$ non-positive with $K_{Y}+\Delta_{Y}$ nef, where $\Delta_{Y}=\phi_{*}\Delta$. As $\phi$ is non-positive $(Y,\Delta_{Y})$ is always lc and if $(X,\Delta)$ is klt then so is $(Y,\Delta_{Y})$. 
		
		If in fact $\phi$ is $K_{X}+\Delta$ negative, $Y$ is $\mathbb{Q}$-factorial, and $(Y,\Delta_{Y})$ is dlt then $\phi$ is a log minimal model. Again if $(X,\Delta)$ is dlt then the dlt condition on $(Y,\Delta_{Y})$ is automatic as $\phi$ is negative.
		
		If instead $\phi:X \dashrightarrow Y$ is a rational map then it is an ample model for $D$ if there is $H$ ample on $Y$ such that $p^{*}D\sim_{\mathbb{R}}q^{*}H+E$ where $E \geq 0$ is such that $E \leq B$ for any $p^{*}D \sim_{\mathbb{R}} B \geq 0$.
	\end{definition}


	The definitions of various birational models for klt or lc pairs in \autoref{Model-defs} extend naturally to the rlt case.

	\begin{definition}
		Let $\phi:X \dashrightarrow Y$ be a rational map. If $U_{i}$ is an open cover of $T$ we write $\phi_{i}:X_{i}\dashrightarrow Y_{i}=Y\times U_{i}$.
		If $(X,\Delta)$ is a psuedoeffective rlc pair witnessed by $(X_{i},\Delta_{i})$ then $\phi$ is a weak log canonical (wlc) model of $(X,\Delta)$ if $\phi_{i}$ is an $(X,\Delta_{i})$ wlc model for each $i$. Equally if $(X,\Delta)$ is rlt then $\phi$ is a log minimal model of $(X,\Delta)$ if and only if each $\phi_{i}$ is a log minimal model of $(X_{i},\Delta_{i})$.
	\end{definition}
	
	By Lemma \ref{equiv} these definitions are independent of the choice of witnesses. In particular if $(X,\Delta)$ is lc then the definition of wlc models agrees with usual one, equally if it is klt then the definition of log minimal model is unchanged.
	
	\begin{remark}
		The usual definition of ample model works here with no modification, it is equivalent to asking for it to be an ample model for the witnesses.
	\end{remark}
	
	\begin{lemma}\label{bertini}
		Let $(X,\Delta)/T$ be an rlt $R$-pair. Take $A\geq 0$ big and nef, then $(X,\Delta+A)$ is rlt. Moreover if $D$ is a divisor on $X$ sharing no components with the augmented base locus $\BS(A)$ nor any witness of $(X,\Delta)$ then we may assume no witness of $(X,\Delta+A)$ shares a component with $D$.
	\end{lemma}
	\begin{proof}
		Write $A \sim A'+E$ for $A'$ ample and $E \geq 0$. We may assume $E$ is arbitrarily small, by writing $A\sim \delta A' + (1-\delta)A=\delta E$ and replacing $A'$ with $\delta A' + (1-\delta)A$. Thus we may suppose $(X,\Delta+E)$ is rlt such that no witnesses shares a component with $D$ and reduce to the case $A$ is ample.
		
		Pick a point $P \in T$ and localise. Write $X_{P}=X \times T_{P}$, $\Delta_{P}$ for the witness over $P$ and $D_{P}$ for the restriction of $D$. Let $\pi:Y \to X_{P}$ be a log resolution of $(X_{P},\Delta_{p}+D)$. Let $D'=\text{Supp}(\pi^{-1}_{*}D)$ and take $-$ effective, exceptional and anti-ample over $X_{p}$. So $A'=\pi^{*}A_{p}-E$ is ample. Write $K_{Y}+\Delta'=\pi^{*}(K_{X_{p}}+\Delta)$.
		
		By \cite[Theorem 2.11]{bhatt2020globally+} we can choose $A'\geq 0$ with $(Y,\Delta'+A'+E)$ klt and $(Y,\Delta'+A'+E+D')$ lc. In particular this choice of $A'$ cannot share a component with $D'$. Now $(X_{P},\Delta_{P}+\pi_{*}A')$ is klt and $\pi_{*}A'$ shares no components with $D$. Then this pair lifts to klt pair over some neighbourhood of $p$. The result follows by quasi-compactness.
	\end{proof}

	The MMP for these pairs lifts naturally from the klt case.
	
	
	\begin{theorem}[rlt Cone Theorem]
		Let $(X,\Delta)$ be an rlt $\mathbb{Q}$-factorial threefold pair $R$-pair with $\mathbb{R}$ boundary. Then there is a countable collection of curves $\{C_{i}\}$ on $X$ such that:
		\begin{enumerate}
			\item $$\overline{NE}(X/U)=\overline{NE}(X/U)_{K_{Y}+\Delta \geq 0} + \sum_{i} \mathbb{R}[C_{i}]$$
			\item The rays $C_{i}$ do not accumulate in $(K_{Y}+\Delta)_{<0}$.
			\item There is an integer $M$ such that for each $i$ there is $d_{C_{i}}$ with 
			\[0 < -(K_{X}+\Delta).C_{i} \leq Md_{c_{i}}\]
			and $d_{C_{i}}$ divides $L\cdot_{k}C_{i}$ for every Cartier divisor $L$ on $X$.
		\end{enumerate}
	\end{theorem}
	
	
	\begin{proof}
		
		For ease of notation we will often view cycles on $X_{i}$ as cycles on $X$ without renaming.
		
		Suppose that $(X,\Delta)$ has witnesses $(X_{i}=X \times U_{i}, \Delta_{i})$ for some open cover $U_{i}$ of $T$. Then $U_{i}$ is still quasi-projective over $R$ and the Cone Theorem holds for each $(X_{i},\Delta_{i})$. Let $\gamma_{i,j}$ be the $K_{X_{i}}+\Delta_{i}$ negative extremal curves. These are also $K_{X}+\Delta$ negative, though they need not be extremal on $X$.
		
		Suppose now that $R$ is a $K_{X}+\Delta$ negative extremal ray. Let $r\in R$ be a non-zero cycle. Then $r$ is the limit of some effective cycles $r^{k}$. In particular if we write $r^{k}_{i}=r^{k}|_{X_{i}}$ then $r_{i}=r|_{X_{i}}=\lim r^{k}_{i}$ is still pseudo-effective. Moreover $r-r_{i}=\lim r^{k}-r^{k}_{i}$ is also psuedo-effective. Since $R$ is extremal we must have for each $i$ that either $r_{i}=0$ or $r=t_{i}r_{i}$ for some $t_{i} > 0$. There must be some $i$ with $r_{i} \neq 0$. However $r_{i}$ then generates an extremal $K_{X}+\Delta$ negative ray, hence $r_{i}=t\gamma_{i,j}$ for some $j$ and some $t>0$. Thus the $\gamma_{i,j}$ generate all the $K_{X}+\Delta$ negative extremal rays. $(a)$ and $(c)$ follow immediately. Since there are finitely many $U_{i}$ if the rays accumulated on $X$ we could chose a subsequence consisting of extremal rays coming from some $X_{i}$ which would then accumulate on $X_{i}$, thus $2$ also holds.
	\end{proof}

	\begin{theorem}[rlt Basepoint Free Theorem]
		Let $(X,\Delta)$ be a $\mathbb{Q}$-factorial threefold rlt $R$-pair with $\mathbb{R}$-boundary. Let $L$ be a nef Cartier divisor over $T$ such that $L-(K_{X}+\Delta)$ is big and nef over $T$. Then $L$ is semiample.
	\end{theorem}
	
	\begin{proof}
		This is immediate from the klt case, \cite{bhatt2020globally+}[Theorem 9.26], since semi-ampleness is local on the base and if $L-(K_{X}+\Delta)$ is big and nef over $T$ then $L_{X_{i}}-(K_{X_{i}}+\Delta_{i})$ is big and nef over $U_{i}$ for each $i$.
	\end{proof}

	\begin{theorem}[Existence of rlt flips]
	Let $(X,\Delta)$ be a threefold rlt $R$-pair with $\mathbb{R}$-boundary. Suppose $X \to Y$ is a flipping contraction over $T$ then the flip $X \dashrightarrow X^{+}$ exists. 
	\end{theorem}

	\begin{proof}
	Let $\phi:X \to Y$ be a flipping contraction for an rlt pair $(X,\Delta)$. Suppose $(X,\Delta)$ is witnessed by $(X_{i},\Delta_{i})$ and let $\phi_{i}:X_{i} \to Y_{i}$ be the induced morphism $U_{i}$. Then $\phi_{i}$ is either still a flipping contraction or an isomorphism. If $\phi_{i}$ is a flipping contraction, then the existence of flip $X_{i}^{+}$ is ensured by \cite{bhatt2020globally+}[Theorem 9.12], otherwise we take simply take $X_{i}^{+}=X_{i}$. Hence we have a suitable $X_{i}^{+}$ for each $i$. Since flips are unique these $X_{i}^{+}$ glue to a variety $X^{+}$ over $T$ such that $X \dashrightarrow X^{+}$ is the required flip.
	\end{proof}
	
	\begin{theorem}[Termination of rlt flips]
		Let $(X,\Delta)$ be a threefold rlt $R$-pair with $\mathbb{R}$-boundary. Then any sequence of $(K_{X}+\Delta)$ flips terminates.
	\end{theorem}
	
	\begin{proof}
		Let $f^{i}:X^{i} \to X^{i+1}$ be a sequence of flips from $X=X^{0}$ of an rlt pair $(K_{X}+\Delta)$. Then $(K_{X}+\Delta)$ is witnessed over some finite open cover $U_{j}$ and the restriction $f^{i}_{j}:X_{j}^{i} \to X_{j}^{i+1}$ is a sequence of flips for the klt pair $(K_{X_{j}}+\Delta_{j})$ for each $j$. In particular for fixed $j$ the sequence eventually terminates by Corollary \ref{termination}, but then as there are finitely many $j$, the global sequence $f^{i}$ also terminates. 
	\end{proof}

\begin{theorem}[MMP for rlt pairs]\label{rltmmp}
	Let $(X,\Delta)$ be a threefold rlt $R$-pair with $\mathbb{R}$-boundary, then we can run a $K_{X}+\Delta$ MMP and any such MMP terminates. If $K_{X}+\Delta$ is pseudo-effective then this terminates with a log minimal model, otherwise it ends in a Mori Fibre Space. If $\Delta$ is big and the output is a log minimal model, then it is a good minimal model.
\end{theorem}


\begin{proof}
	The first part is immediate from the above results. Suppose then $\phi:X \dashrightarrow Y$ is a log minimal model and $\Delta$ is big. If $\Delta$ is a big $\mathbb{Q}$ divisor then $\phi$ is good by the Basepoint Free Theorem. 
\end{proof}

	\section{RLT Polytopes}
	
	\begin{definition}
		
		Fix a $\mathbb{Q}$-divisor $A\geq 0$. Let $V$ be a finite dimensional, rational affine subspace of $WDiv_{\mathbb{R}}(X)$ containing no components of $A$. Such $V$ is called a coefficient space (for A).
		
		We have the following.
		\[V_{A}= \{A+B: B \in V\}\]
		\[\mathcal{L}_{A}(V)=\{\Delta=A+B \in V_{A}: (X,\Delta) \text{ is a klt pair}\}\]
		\[\mathcal{RL}_{A}(V)=\{\Delta=A+B \in V_{A}: (X,\Delta) \text{ is an rlc pair with witnesses in } V_{A}\}\]
		%\[\mathcal{RE}_{A}(V)=\{\Delta \in \mathcal{RL}_{A}(V): K_{X}+\Delta \text{ is pseudoeffective}\}\]
		%\[\mathcal{RN}_{A}(V)=\{\Delta \in \mathcal{RL}_{A}(V): K_{X}+\Delta \text{ is nef}\}\]
		
		We call a polytope $C$ inside $\mathcal{RL}_{A}(V)$ rlt if it is rational and contains only boundaries of rlt pairs.
		
		If $C \subseteq \mathcal{RL}_{A}(V)$ is a rational polytope then we have
		\[\mathcal{E}(C)=\{\Delta \in C: K_{X}+\Delta \text{ is pseudoeffective}\}\]
		\[\mathcal{N}(C)=\{\Delta \in C: K_{X}+\Delta \text{ is nef}\}\]
		
		Given a birational contraction $\phi:X \dashrightarrow Y$ we also define
		%\[\mathcal{RW}_{A,\phi}(V)=\{\Delta \in \mathcal{RE}_{A}(V): \phi \text{ is a weak log canonical (wlc) model of } (X,\Delta)\}\]
		\[\mathcal{W}_{\phi}(C)=\{\Delta \in \mathcal{E}(C): \phi \text{ is a weak log canonical (wlc) model of } (X,\Delta)\}\]
		and given a rational map $\psi:X \dashrightarrow Z$
		%\[\mathcal{RA}_{A,\phi}(V)=\{\Delta \in \mathcal{RE}_{A}(V): \phi \text{ is the ample model of } (X,\Delta)\}\]
		\[\mathcal{A}_{\phi}(C)=\{\Delta \in \mathcal{E}(C): \phi \text{ is the ample model of } (X,\Delta)\}\]
	\end{definition}
	
	
	\begin{remark}
		
		As defined above, $\mathcal{RL}_{A}(V)$ is non-empty only when $(X,A)$ is log canonical. We might wish to allow $(X,A)$ to be rlc with fixed witnesses instead. This quickly becomes non-trivial because of the overlap of sets in the corresponding open cover.
		
		If we're interested in a pair $(X,A+B)$ where $(X,B)$ is rlt and $A$ is big and nef then for suitably small $t>0$ we always have that $(X,tA+(1-t)A+B)$ is rlt with coefficients in $\mathcal{RL}_{tA}(V)$ by Lemma \ref{bertini}. Moreover if we have finitely many such pairs, we can find $t$ suitable for all of them. This is normally enough in practice.
		
	\end{remark}
	
	\begin{lemma}
		Take $A \geq 0$ and let $V$ be a coefficient space. Let $C \subseteq \mathcal{RL}_{A}(V)$ be a rational polytope. Then there is an open cover $U_{i}$ such that every $\Delta \in C$ is witnessed over $U_{i}$. If $C$ is an rlt polytope then we may choose $U_{i}$ such that every witness is klt.
	\end{lemma}
	\begin{proof}
		We can take the vertices $D_{i}$ of $C$. Then take witnesses $(X_{i,j}, B_{i,j})$ of $D_{i}$. Since there are finitely many $D_{i}$, we can assume that for all $i$ we have $X_{i,j}=X_{j}$ for some $X_{j}$ not depending on $i$, after taking intersections of combinations of the $X_{i,j}$ and renumbering as necessary. Now $C$ is the convex hull of the $D_{i}$ and $\Delta= \sum \lambda_{i}D_{i}$ has witnesses $\Delta_{j}= \sum \lambda_{i}B_{i,j}$ as required.
	\end{proof}
	
	We will essentially only ever work with rational polytopes containing a klt boundary. Since the questions are always local we can normally assume these polytopes are simplices. By the following lemma, it is then enough to work with rlt polytopes.
	
	\begin{lemma}\label{rlt-repl}
		Suppose $A$ is ample, $V$ is a coefficient space and that $C\subseteq \mathcal{RL}_{A}(V)$ is a rational simplex. If there is some boundary $B_{0} \in \mathcal{RL}_{A}(V)$ with $(X,B_{0})$ rlt, then there is an affine bijection $f:C \to C'$, where $C'$ is an rlt polytope inside $\mathcal{RL}_{A/2}(W)$ for some coefficient space $W$. Further $f, f^{-1}$ preserve rationality and $\mathbb{Q}$-linear equivalence.
	\end{lemma}

	\begin{proof}
		To show a rational polytope $C'\subseteq \mathcal{RL}_{A'}(V')$ is rlt it is enough to show that every vertex boundary $B_{i}$ of $C'$ is rlt with witnesses in $V'$.
				
		Indeed if this is the case then for $B \in C$ we have $B= \sum \lambda_{i} B_{i}$ for $0 \leq \lambda_{i} \leq 1$. Let $U_{j}$ be an open cover such that each $B_{i}$ is witnessed by $(X_{j},B_{i,j})$, then $B|_{X_{j}}\sim \sum \lambda_{i}B_{i,j}$, so $(X,B)$ must be rlt as claimed.
		
		
		
		Write the vertices of $C$ as $B_{i}=A+\Delta_{i}$ for $i > 0$ and let be $B_{0}=A+\Delta_{0} \in \mathcal{RL}_{A}(V)$ be the rlt boundary. Now choose $\Gamma_{i} =(1-t_{i})\Delta_{i}+t_{i}\Delta_{0}$ for $t_{i}$ rational and sufficiently small that $\frac{A}{2}+t_{i}(\Delta_{i}-\Delta_{0})$ is ample. By construction $(X,A+\Gamma_{i})$ is rlt.
		
		Further choose $H_{i} \sim_{\mathbb{Q}} \frac{A}{2}+t_{i}(\Delta_{i}-\Delta_{0})$ effective and sharing no support with $A$. Then by construction
		\[A+\Delta_{i} \sim_{\mathbb{Q}} \frac{A}{2}+\Gamma_{i}+H_{i}=D_{i}\]
		and $(X,D_{i})$ is rlt by Lemma \ref{bertini}. Reselecting $H_{i}$ if needed we may suppose that $D_{i}$ is not in the span of $\{D_{j}: i \neq j\}$ for each $i$. This can always be done since the $H_{i}$ are all ample.
		
		Let $W$ be a coefficient space containing the components of $\Delta_{i}, H_{i}$ such that each $(X,D_{i})$ is rlt with witnesses in $W$. Now let $C'$ be the convex hull of the $D_{i}$, so that $C'$ is an rlt polytope inside $\mathcal{RL}_{A}(W)$.
		
		Since $C$ is a simplex, by assumption, we can write any $B \in C$ uniquely as $B=\sum \lambda_{i} B_{i}$ where $\lambda_{i} \geq 0$ and $\sum \lambda_{i} =1$. Therefore, we can define a bijective affine map $f: C \to C'$ by sending $B_{i}=A+\Delta_{i} \to D_{i}$ and then writing $f(B)= \sum \lambda_{i} D_{i}$.
		
		Clearly $B$ is rational if and only if $\lambda_{i} \in \mathbb{Q}$, which happens if and only if $f(B)=\sum \lambda_{i} D_{i}$ is rational. So $f, f^{-1}$ preserve rationality. Equally as $B_{i} \sim_{\mathbb{Q}} D_{i}$ we must have $B \sim_{\mathbb{Q}} f(B)$, and the same holds for $f^{-1}$.
			
	\end{proof}

\begin{remark}
	With the notation of Lemma \ref{rlt-repl}, if $S \subseteq C$ is a rational polytope then $f(S)$ is also a rational polytope since $f$ is affine and preserves rationality. The converse is also true since $f^{-1}$ is also still affine and $f^{-1}f(S)=S$ as $f$ is a bijection.  
\end{remark}


	Given a general rlc polytope we can always take a rational triangulation and define a piecewise affine bijection, $f$, by using the above procedure on each simplex. However, this does not in general preserve convexity, so it easier in practice to work locally on the polytope and assume it is a simplex. Alternatively, this could be remedied by working with $C'$, the convex hull of $f(C)$, since this must still be an rlt polytope. Then $f\colon C \to C'$ is no longer a bijection, but it is still preserves rationality and $\mathbb{Q}$-linear equivalence so would suffice for applications. 

	
	\begin{definition}
		Take $S, S' \subseteq \mathcal{RL}_{A}(V)$. We say $S \sim_{\mathbb{R}} S'$ if for every $\Delta \in S$ there is $\Delta' \in S'$ with $\Delta \sim_{\mathbb{R}} \Delta'$ and vice versa. The linear closure of $S$ is given by $$S^{*}=\bigcup_{S' \sim S}S'= \{\Delta \in \mathcal{RL}_{A}(V) \text{ such that } \exists \Delta' \in S \text{ with }\Delta \sim_{\mathbb{R}} \Delta'\}$$.
	\end{definition}
	
	

	\begin{lemma}
		Let $V$ be a finite dimensional, rational affine subspace of $WDiv_{\mathbb{R}}(X)$ and fix $A \geq 0$. Take $S \subseteq \mathcal{RL}_{A}(V)$ a rational polytope. Then the linear closure, $S^{*}$ is also a rational polytope. 
	\end{lemma}
	
	\begin{proof}
		By translating by $-A$ we can view $S$ as a subset of $V$. Similarly, after a translation by say $D$ of $V$ we can suppose that $V$ is a vector space. After these transformations we have that $S^{*}=\{B+E \text{ such that } B\in S, E \sim 0 \text{ and } B+E -D \geq 0\}$.
		
		
		Let $N=\{E \in V: E \sim_{\mathbb{R}} 0\}$ and take $\phi:V \to W=V/N \subseteq \text{Pic}(X)\otimes \mathbb{R}$, then $\phi(S)=\phi(S^{*})$ is a rational polytope in $W$ and its preimage $S+N$ is still cut out by finitely rational half spaces, but is no longer compact. Hence we must have that $S^{*}=(S+N)\cap({\Delta \geq D})$ is cut out by finitely many rational half spaces. 
		
		However for each point $B \in S$, the set $\{B\}^{*}=\{B+E\geq D \text{ such that } E \sim_{\mathbb{R}} 0\}$ is bounded, since the $E\in N$ such that $B+E \geq D$ are bounded by the coefficients of $B$ and $D$. Since $S$ is closed and bounded however we must have that $S^{*}$ is bounded too.
	\end{proof}
	

	
	In particular $\mathcal{RL}_{A}(V)$ is a rational polytope over a local ring, since it is the linear closure of $\mathcal{L}_{A}(V)$. To lift from the local case, we essentially find an open cover of $T$ which witnesses $\mathcal{RL}_{A}(V)$.
	
	\begin{theorem}\label{rlt-poly}
		Let $V$ be a finite dimensional, rational affine subspace of $WDiv_{\mathbb{R}}(X)$ and fix $A \geq 0$. Then $\mathcal{RL}_{A}(V)$ is a rational polytope.
	\end{theorem}
	
	\begin{proof}
		For $W$ an affine subspace, let $\hat{W}=\{w-w' \text{ such that } w,w' \in W\}$.
		
		Take a point $p \in T$, and consider $X_{p}=X\times T_{p} \to T_{p}$. Let $A_{p},V_{p}$ be the restrictions of $A,V$ to $X_{p}$ and let $D_{i}$ be the vertices of $\mathcal{L}_{A_{p}}(V_{p})$, then there are open sets $U_{i}$ around $p$ such that $(X\times U_{i},D_{i})$ are lc when $D_{i}$ is extended over $U_{i}$. Moreover we may freely assume that there are no vertical components of $V$ which meet $U_{p}= \bigcap U_{i}$ but are not supported over $p$, thus ensuring for $E$ in $\hat{V}|_{X_{U_{p}}}$ where $X_{U}=X\times U_{p}$, we have $E \sim_{\mathbb{R}} 0$ if and only if $E|_{X_{p}}\sim_{\mathbb{R}} 0$. By compactness of $T$ there are finitely many $p_{j}$ such that $U_{j}=U_{p_{j}}$ is an open cover of $T$. 
		
		A pair $(X,\Delta)$ is rlc if and only if it is witnessed over $U_{j}$. Indeed if it is rlc, then we must be able to find $B_{j}$ such $(X_{p_{j}},B_{j})$ is lc and $B_{j} \sim_{\mathbb{R}} \Delta$. By construction however $B_{j}$ extends to an lc pair $(X_{j}=X\times U_{p_{j}},B_{j})$. Then $(X,\Delta)$ is witnessed by $(X_{j}, B_{j})$ as required.
		
		Consider $\mathcal{RL}_{A}(V)$, by the previous paragraph we may take an open cover $U_{i}$ such that every pair $(X,B) \in \mathcal{RL}_{A}(V)$ is witnessed by pairs $(X_{i}=X\times U_{i},B_{i})$. Let $C_{i} = \mathcal{L}_{A_{i}}(V_{i})^{*}$ where $A_{i}, V_{i}$ are the restrictions of $A,V$ to $X_{i}$ and write $S_{i}=\{\Delta \in V: \Delta|_{X_{i}} \in C_{i}\}$, then $\mathcal{RL}_{A}(V)= \bigcap S_{i}$ is a rational polytope since each $C_{i}$ is and there are no divisors $D \neq 0$ with $D|_{X_{i}} \neq 0$ for every $i$.
	\end{proof}
	
	In particular then $\mathcal{RL}_{A}(V)$ is closed. Moreover since it is a polytope, if $(X,\Delta_{i})$ is a sequence of rlc pairs with $\Delta_{i} \to \Delta$, then the witnesses of $\Delta$ may be chosen to be the limit of witnesses of $\Delta_{i}$. 
	
	We also proved that there was a finite open cover witnessing every $B \in \mathcal{RL}_{A}(V)$. If $C \subseteq \mathcal{RL}_{A}(V)$ is an rlt polytope then we may choose this cover such that every witness for $C$ is klt (rather than just lc). 

	\section{Finiteness of Log Terminal Models}
	
%	In this section $A$ is always an ample $\mathbb{Q}$-divisor on $X$ unless otherwise specified and $V$ is a coefficient space. All the results about rlt polytopes in $\mathcal{RL}_{A}(C)$ extend to the case that $A$ is big and nef, by replacing it with $A'+E$ where $A$ is ample and $E$ is suitably small. This is not true for results about general rational (i.e. rlc) polytopes however. \myworries{Is this true?}
	
	\begin{lemma}\label{neftope}
		Fix a $\mathbb{Q}$-divisor $A \geq 0$ and let $C\subseteq \mathcal{L}_{A}(V)$ be a rational polytope. Then $\mathcal{N}(C)=\{\Delta \in C \text{ such that } K_{X}+\Delta \text{ is nef } \}$ is also a rational polytope.
	\end{lemma}

	\begin{proof}	
		Let $B_{i}$ be the vertices of $C$. If $B \in C$ then $B= \sum \lambda_{i} B_{i}$ for $1 \geq \lambda_{i} \geq 0$ so $(K_{X}+B).C <0$ ensures $(K_{X}+B_{i}).C <0$ for some $i$. In particular if $R_{i,j}$ are the $K_{X}+B_{i}$ negative extremal rays then $K_{X}+B$ is nef if and only if $(K_{X}+B).R_{i,j} \geq 0$ for all $i,j$. Indeed, suppose that we have such a $K_{X}+B$ and that $R$ is a $K_{X}+B$ negative extremal ray, then $(K_{X}+B_{i}).R <0$ for some $i$ and so $R=R_{i,j}$ for some $j$, a contradiction. Then the condition $(K_{X}+B).R_{i,j} \geq 0$ defines a rational polytope by \cite[Proposition 9.31]{bhatt2020globally+}.
	\end{proof}
	
	Since this result does not require $A$ to be ample, we may often avoid the use of Bertini's Theorem, \cite[Lemma 3.7.3]{birkar2010existence} in particular, to substitute a big divisor for an ample one. Versions of these results are available for rlt polytopes but making use of them requires extra back and forth between the klt and rlt case. 
	
	\begin{lemma}\label{rationality}
		Let $\phi: X \dashrightarrow Y$ be a birational contraction. Let $C \subseteq \mathcal{RL}_{A}(V)$ be an rlt polytope, then $\mathcal{W}_{\phi}(C)$ is a rational polytope.
	\end{lemma}
	\begin{proof}
		We can choose a finite open cover, $U_{i}$ such that $C$ is witnessed by klt pairs over $U_{i}$. On $X_{i}$ we can write $N_{i}=\{E\sim_{\mathbb{R}} 0\} \subseteq V_{i}=V|_{X_{i}}$, $C_{i}=C|_{X_{i}}$ and consider the induced map $\phi_{i}:X_{i} \to Y_{i}$. Now let $C'_{i}=\mathcal{L}_{A_{i}}(V_{i}) \cap C_{i}^{*}$. After perhaps shrinking $C'_{i}$ we may suppose it is a klt polytope and $C_{i} \subseteq (C')^{*}_{i}$. Thus $\mathcal{W}_{\phi_{i}}(C'_{i})$ is a rational polytope by \cite[Corollary 3.11.2]{birkar2010existence} with \cite[Theorem 3.11.1]{birkar2010existence} and \cite[Lemma 3.7.4]{birkar2010existence} replaced by Lemma \ref{neftope}.
		
		Therefore $\mathcal{W}_{i}=\mathcal{W}_{\phi_{i}}(C_{i})=\mathcal{W}_{\phi_{i}}(C'_{i})^{*}\cap C_{i}$ is also a rational polytope. For each $\mathcal{W}_{i}$ we have a rational polytope $\hat{\mathcal{W}_{i}}=\{\Delta \in C: \Delta|_{X_{i}} \in \mathcal{W}_{i}\} \subseteq C$. The intersection of these polytopes is precisely $\mathcal{W}_{\phi}(C)$.
	\end{proof}
	
	\begin{lemma}\label{faces}
		Let $\phi: X \dashrightarrow Y$ be a birational contraction. Let $C$ be an rlt polytope, let $F \subseteq \mathcal{W}_{\phi}(C)$ be a face, possibly with $F =\mathcal{W}_{\phi}(C)$. Suppose $f: X \dashrightarrow Z$ is an ample model for some $B$ in the interior of $F$. Then there is a factorisation $f=g \circ \phi$ for some morphism $g:Y \to Z$, and moreover $f$ is an ample model for every boundary in the interior of $F$.
	\end{lemma}
	
	\begin{proof}
		
		Since $\phi$ is a wlc model for $B$ we have an induced map $g: Y \to Z'$. However then $g\circ \phi$ is an ample model for $(X,B)$, so after post-composition with an isomorphism we may suppose $Z=Z'$ and $f=g\circ \phi$. Suppose $B' \in \mathcal{W}_{\phi}(C)$ then $f$ is an ample model for $(X,B')$ if and only $g$ is an ample model for $(Y,\phi_{*}B')$. Since $K_{Y}+\phi_{*}B$' is semiample $g$ is an ample model if and only if the curves contracted by $g$ are precisely those $\Gamma$ with $(K_{Y}+\phi_{*}B').\Gamma=0$.
		
		Suppose then $B'$ is in the interior of $F$. Consider $B_{t}=tB+(1-t)B'$, so that $K_{Y}+\phi_{*}B_{t}= t(K_{Y}+\phi_{*}B)+(1-t)(K_{Y}+\phi_{*}B')$. Then if $(K_{Y}+\phi_{*}B').\Gamma \neq 0$ then and $(K_{Y}+\phi_{*}B).\Gamma=0$ it must be that $(K_{Y}+\phi_{*}B_{t}).\Gamma < 0$ for all $t < 0$. However for small $t$ we have $B_{t} \in F$, a contradiction. By symmetry, we see that $\Gamma$ is contracted by $g$ if and only if $(K_{Y}+\phi_{*}B').\Gamma=0$, so $f$ is an ample model for $K_{X}+\phi_{*}B'$ also.
	\end{proof} 


	\begin{theorem}\cite[Theorem 9.34]{bhatt2020globally+}
		Let $C$ be a klt polytope in $\mathcal{L}_{A}(V)$ for $A$ big. There is a finite collection of log terminal models $\phi_{i}: X \dashrightarrow Y_{i}$ such that  every $B \in \mathcal{E}(C)$ has some $j$ with $\phi_{j}$ a log terminal model of $(X,B)$. 
	\end{theorem}
	

	
	\begin{corollary}\label{klt_finiteness}
		Let $C$ be a klt polytope with $A$ big. Suppose that every $B \in C$ has components which span $NS(X)$, then there are finitely many birational maps $\phi_{i}: X \dashrightarrow Y_{i}$ such that for any $B \in \mathcal{E}(C)$ if $\phi: X \dashrightarrow Y$ is a wlc model then $\phi_{i}=f \circ \phi$ for some $i$ and some isomorphism $f: Y \to Y_{i}$.
	\end{corollary}
	
	\begin{proof}
		
		Take $C'\subseteq \mathcal{L}_{A}(V)$ a klt polytope with $C \subseteq C'$ such that for any $B \in C$ if $D$ is a component of $B$ then $B+tD$ is in $C$ for $|t| > \epsilon$, for some $\epsilon >0$ depending on $B$ and $D$. This can be done by taking $C'$ to be the convex hull of small perturbations of the vertices of $C$.
		
		By the previous theorem there are finitely many birational maps $\phi_{i}: X \dashrightarrow Y_{i}$ such that for every $B \in \mathcal{E}(C')$ there is some $\phi_{i}$ a log minimal model of $(X,\Delta)$.
		
		Further are then finitely many morphisms $f_{i,j}:Y_{i}\to Z_{j}$ such that $\psi_{i,j}=f_{i,j} \circ \phi_{i}$ are ample models such that $B \in \mathcal{E}(C')$ some $\psi_{i,j}$ is the (unique) ample model of $(X,B)$. This is because the $f_{i,j}$ correspond to faces of the rational polytope $\mathcal{W}_{\phi_{i}}(C')$ by Lemma \ref{faces}.
		
		Now pick $\Delta \in C$. Let $\psi: X \dashrightarrow Y$ be a wlc for $\Delta$. We can take $D$ in the span of the components of $B$ such that $\phi$ is $B+D$ negative and $\phi_{*}D$ is ample. By shrinking $D$, we can suppose that $B+D\in C'$. Thus we have that $\psi$ is the ample model of some $B+D \in \mathcal{W}_{\psi}(C')$. Now take a log terminal model of $B+D$ of the form $\phi_{i}$ for some $i$, then up to post-composition with an isomorphism we have $\psi=f_{i,j} \circ \phi_{i}=\psi_{i,j}$ for some $j$. 
		
		Thus the family of models $\{\psi_{i,j}\}$ give the required maps.
		
	\end{proof}


	\begin{theorem}\label{weak finiteness}
		Let $A$ be big and nef and chose $V$ a coefficient space. Take $C$ be an rlt polytope inside $\mathcal{RL}_{A}(V)$, then
		
		\begin{enumerate}
			\item There are finitely many birational maps $\phi_{j}: X \dashrightarrow Y_{j}$ such that for any $B \in \mathcal{E}(C)$ if $\phi: X \dashrightarrow Y$ is a wlc model then $\phi_{j}=f \circ \phi$ for some $j$ and some isomorphism $f: Y \to Y_{j}$. \\
			\item There are finitely many rational maps $\psi_{k}: X \dashrightarrow Z_{k}$ such that if $\psi:X \dashrightarrow Z$ is an ample model for some $B \in \mathcal{E}(C)$ then there is an isomorphism $f:Z \to Z_{k}$ with $\psi_{k}=f \circ \psi_{k}$.
		\end{enumerate}
	\end{theorem}
	
	\begin{proof}
		We prove 1., 2. follows immediately as ample models correspond to the interiors of faces of the $\mathcal{W}_{\phi_{i}}(C)$ by Lemma \ref{faces}.\\	
		
		Equally, it is enough to show this in the case that $C$ is a klt polytope. Indeed suppose it holds for klt polytopes. Then take an open cover $U_{i}$ of $T$ witnessing $C$. For each $i$ we may take a klt polytope $C'_{i}$ with $C'_{i} \sim C_{i}=C|_{U_{i}}$. Given a wlc map $\phi:X \dashrightarrow Z$ for $B \in \mathcal{E}(C)$, we can let $\phi_{i}$ be the induced map on $X_{i}$ which is a wlc model for some $B_{i} \in C'_{i}$. In particular for fixed $i$ there are finitely many $\phi_{i,j}$ such that for any $B$ and $\phi$ we have $f_{i} \circ \phi_{i}=\phi_{i,j}$ for some $j$ and $f_{i}$. As $U_{i}$ is a finite cover there are finitely many $\phi_{i,j}$ indexed over $i,j$.
		
		If we have another map $\Phi:X \dashleftarrow Z'$ with isomorphisms $g_{i}$ such that $f_{i} \circ \phi_{i}=\phi_{i,j}= g_{i} \circ \Phi_{i}$, then $h_{i}=g_{i} \circ f^{-1}_{i}$ glues to an isomorphism $Z' \to Z$ over $T$. Thus there are only finitely many wlc models up to isomorphism.
		
		Suppose then that $C$ is a klt polytope.		
		
		Let $\pi:Y \to X$ be a log resolution of the support of $V$. Then for any $\Delta$ in $C$ we have $\pi^{*}(K_{X}+\Delta)+E=(K_{Y}+\Delta')$ where $E \geq $ is exceptional and shares no components with $\Delta'$ and $(Y,\Delta')$ is klt. Sending $\Delta \to \Delta'$ as above we can find a new polytope $C'$ on which it is sufficient to check the result holds. By replacing $C$ with $C'$, $A$ with $\pi^{*}A$, $X$ with $Y$ and $V$ with a suitable space, we may suppose that $X$ is smooth, though it may no longer be the case that $A$ shares no support with $V$. 
		
		Let $H_{k}$ be ample divisors spanning $NS(X)$. Let $H= \sum H_{k}$. Note that for any open $U$ in $T$ we still have the components of $H|_{X_{U}}$ span $NS(X_{U})$, since $NS(X)$ surjects on $NS(X_{U})$.
		
		After shrinking $H$ we may suppose that $A \sim A'+E$, for $A'$ ample and sharing no components with $V,H$ or $E$ with $A-H$ ample and $E \geq 0$ such that $$\{H+B+E \colon A+B \in C\}$$ is klt. Further we may then take $t >0$ sufficiently small that $$\{tA+H+B+E \colon A+B \in C\}$$ is klt and thus $$C'=\{tA+(1-t)A+H+B+E \colon A+B \in C\}$$ is rlt by Lemma \ref{bertini}. Thus we may extend $V$ to a coefficient space $W$ such that $C' \subseteq \mathcal{RL}_{tA}(W)$. By taking an open cover as before, we may in fact assume that $C$ is klt. But then the result follows by Corollary \ref{klt_finiteness}, since the components of $H$ span $NS(X)$ by constructions.	
		
	\end{proof}
	
	\begin{theorem}\label{rltfiniteness}
		Let $A$ be a big and nef $\mathbb{Q}$-Cartier divisor and $C$ be a rational polytope inside $\mathcal{RL}_{A}(V)$. Suppose there is a boundary $A+B \in \mathcal{RL}_{A}(V)$ such that $(X,A+B)$ is rlt with witnesses in $V_{A}$. Then the following hold:
		
		\begin{enumerate}
			\item There are finitely many birational contractions $\phi_{i}:X \dashrightarrow Y_{i}$ such that 
			\[\mathcal{E}(C) = \bigcup \mathcal{W}_{i}=\mathcal{W}_{\phi_{i}}(C)\]
			where each $\mathcal{W}_{i}$ is a rational polytope. Moreover if $\phi:X \to Y$ is a wlc model for any choice of $\Delta \in \mathcal{E}(C)$ then $\phi=\phi_{i}$ for some $i$, up to composition with an isomorphism.
			
			\item There are finitely many rational maps $\psi_{j}:X \dashrightarrow Z_{j}$ which partition $\mathcal{E}(C)$ into subsets $\mathcal{A}_{\psi_{j}}(C)=\mathcal{A}_{i}$.
			\item  For each $W_{i}$ there is a $j$ such that we can find a morphism $f_{i,j}: Y_{i} \to Z_{j}$ and $W_{i} \subseteq \overline{A_{j}}$.
			\item  $\mathcal{E}(C)$ is a rational polytope and $A_{j}$ is a union of the interiors of finitely many rational polytopes.
		\end{enumerate}
	\end{theorem}
	
	\begin{proof}
		
		Since the convexity condition of every sub-polytope in the theorem statement is clear, it is enough to show that the result holds for every simplex in a rational triangulation of $C$. Thus after extending $V$ and changing $A$ as needed we may suppose:
		
		\begin{itemize}
			\item $C$ is a simplex 
			\item $C$ is an rlt polytope by Lemma \ref{rlt-repl}
			\item $\mathcal{E}(C)$ is covered by $\mathcal{W}_{\phi_{i}}(C)$  and has a decomposition into disjoint sets $\mathcal{A}_{\psi_{j}}(C)$,
			for some collection of birational contractions $\phi_{i}$ and rational maps $\psi_{j}$ by Theorem \ref{rltmmp} 
			\item There are only finitely many $\phi_{i}$ and $\psi_{j}$ by Theorem \ref{weak finiteness}.
		\end{itemize}
			
		Take one of the wlc models $\phi_{i}:X \dashrightarrow Y_{i}$ , then just as in Lemma $\ref{faces}$, if $\Delta,\Delta'$ are in the same face of $\mathcal{W}_{i}$ then they have the same ample model. In particular then let $\psi_{j}:X \dashrightarrow Z_{j}$ be the ample model corresponding to the interior of $\mathcal{W}_{i}$, then we have a morphism $f_{i,j}: Y_{i} \to Z_{j}$ and  $W_{i} \subseteq \overline{A_{j}}$ as required. 
		
		Similarly by Lemma \ref{faces} we have that $A_{j} \cap \mathcal{W}_{i}$ is a union of the interiors of some faces of $\mathcal{W}_{i}$. Since there are finitely many $\mathcal{W}_{i}$ and they cover $\mathcal{E}(C)$ the result follows.
	\end{proof}

	\begin{remark}
		In practice since we can always extend $V$ and $C$ it is enough to know that $(X,A)$ is klt, rather than needing an rlt pair $(X,A+B)$. Similarly if $X$ is klt, we can always find $t>0$ such that $(X,tA)$ is klt. Then if $(X,A+B)=(X,tA+(1-t)A+B)$ is rlc with coefficients in $V_{A}$ it is also rlc with coefficients in $V'_{tA}$ for some coefficient space $V'$. By choosing $V'$ such that all the vertices of $C$ are rlc with coefficients in $V'_{tA}$, we see that it is enough to suppose that $X$ is klt.
	\end{remark}

	\section{Geography of Ample Models}
	
	We keep the notation of the previous section, though we denote the closure of $\mathcal{A}_{\phi}(C)$ by $\mathcal{D}_{\phi}(C)$. We will say the span of a polytope $C$ is $$\text{Span}(C)=\{\lambda(B-B') \text{ such that } B, B' \in C \text{ and } \lambda \in \mathbb{R}\}.$$  In a slight abuse of notation we say that $C \in WDiv(X)$ spans $NS(X)$ if the span of $C$ surjects onto $NS(X)$. Equivalently this means if $D$ is a divisor and $B$ is in the interior of $C$ then for all sufficiently small $t>0$ $B+tD\equiv D'_{t}$ for some $D'_{t}\in C$.
		
	\begin{lemma}\label{close}
		Let $X$ be a $\mathbb{Q}$-factorial threefold over $R$. Let $\phi:X \dashrightarrow Y$ be a wlc model of an rlt pair $(X,\Delta)$. Let $C$ be an rlt polytope inside $\mathcal{L}_{A}(V)$. Then we have $\overline{\mathcal{A}_{\phi}(C)} \subseteq \mathcal{W}_{\phi}(C)$ is a rational polytope, moreover if $C$ spans $NS(X)$ and contains an open set around $\Delta$ then this inclusion is an equality.
	\end{lemma}
	
	\begin{proof}
		Suppose that $B \in \mathcal{A}_{\phi}(C)$. Then by \cite[Theorem 3.6.5]{birkar2010existence} we see that in fact $\phi$ is a wlc model for $B$ and thus we have $\mathcal{A}_{\phi}(C) \subseteq \mathcal{W}_{\phi}(C)$. So $\overline{\mathcal{A}_{\phi}(C)}$ is a union of faces of $\mathcal{W}_{\phi}(C)$ by Lemma \ref{faces}. However $\mathcal{A}_{\phi}$ is convex inside $\mathcal{W}_{\phi}(C)$ so it must be that $\overline{\mathcal{A}_{\phi}(C)}$ is a face of $\mathcal{W}_{\phi}(C)$, and thus is a polytope.
		
		Now suppose $C$ spans $NS(X)$ and contains an open set around $\Delta$.
		Let $H$ be a general ample divisor on $Y$. Let $W$ be a common resolution with maps $p:W \to X$, $q:W \to Y$. Then by assumption there is some $H' \equiv p_{*}q^{*}H$ with support contained in the support of $\Delta$, and hence in the support of any $B$ in the interior of $\mathcal{W}_{\phi}(C)$. Take such a $B$, then there is $\epsilon >0$ with $(X,B+\epsilon H') \in C$ , for any $\epsilon' \in ((0,\epsilon])$ $\phi$ is an ample model of $(X,B+ \epsilon' H')$, such an $\epsilon$ exists since $\phi$ is necessarily $H'$ negative. Thus $B+\epsilon' H' \in \mathcal{A}_{\phi}(C)$. But then we must have $\mathcal{W}_{\phi}(C)\subseteq \overline{\mathcal{A}_{\phi}(C)}$.
	\end{proof}
	
	\begin{theorem}\label{assumptions}\cite[Theorem 3.3]{hacon2009sarkisov}
		Let $C$ be a polytope inside $\mathcal{RL}_{A}(V)$, then there are finitely many $f_{i}:X \dashrightarrow Y_{i}$ with the following. properties.
		
		\begin{enumerate}
			\item $\{A_{i}=A_{f_{i}}(C)\}$ partition $\mathcal{E}(C)$. If $f_{i}$ is birational then $\mathcal{D}_{i}=\mathcal{D}_{\phi_{i}}(C)$ is a rational polytope.
			\item If $A_{j} \cap \mathcal{D}_{i} \neq \emptyset$ then there is a morphism $f_{i,j}:Y_{i} \to Y_{j}$ such that $f_{i}=f_{i,j} \circ f_{j}$.
		\end{enumerate}
		
		Moreover if $C$ spans $NS(X)$ then we also have the following.
		
		\begin{enumerate}
			\item[3.] Pick $i$ such that a connected component, $\mathcal{D}$ of $\mathcal{D}_{i}$ meets the interior of $C$. Then the following are equivalent:
			\begin{itemize}
				\item $\dim \mathcal{D}= \dim C$
				\item $\text{Span}(\mathcal{D})=\text{Span}(C)$.
				\item If $B \in \mathcal{A}_{i} \cap \mathcal{D}$ then $f_{i}$ is a log terminal model of $(X,B)$.
				\item $f_{i}$ is birational and $X_{i}$ is $\mathbb{Q}$-factorial
			\end{itemize} 
			
			\item[4.] Suppose that $\mathcal{D}_{i}$ has the same span as $C$ and $B$ is a general point in $\mathcal{A}_{j} \cap \mathcal{D}_{i}$. If in fact $B$ is in the interior of $C$ then the relative picard number of $Y_{i}/Y_{j}$ is the difference in dimension of $\mathcal{D}_{i}$ and $\mathcal{D}_{j} \cap \mathcal{D}_{i}$. 
			
		\end{enumerate}
	\end{theorem}
	
	This result is stated for $C=\mathcal{L}_{A}(V)$ in characteristic zero, but the proof goes through essentially verbatim in this setting.
	
	For brevity we fix some notation, essentially due to Shokurov.
	
	\begin{definition}
		Take a coefficient space $V$, an ample divisor $A$ and then let $C$ be an rlt polytope inside some $\mathcal{RL}_{A}(V)$.
		
		Suppose that $3$ and $4$ of the previous lemma hold for $C$, then the triple $(C,A,V)$ is a said to be a geography, when $A$ and $V$ are clear we sometimes just call $C$ a geography. The dimension of $(C,A,V)$ will be the dimension of $C$.
		The $\mathcal{D}_{\phi}$ are called classes.
		If $C$ is a geography and $\dim \mathcal{D_{\phi}}= \dim C$ then $\mathcal{D}_{\phi}$ is said to be a country. The codimension $1$ faces of countries are called borders, and a codimension $2$ face is called a ridge.
		If $(X,B)$ is a pair such that every country in $C$ is induced by a log minimal model of $(X,B)$ then $(C,A,V)$ is a geography for $(X,B)$.
	\end{definition}

	Theorem \ref{assumptions} then says that if $(C,A,V)$ is a triple such that $C$ spans $NS(X)$ then $C$ is a geography.
	This combined with following will be the main method of producing geographies for the remainder of the section.
	
	\begin{lemma}
		Let $(C,A,V)$ be a geography. Take $W \subseteq V$ be a general coefficient space and let $W_{A}=\{A+B, B \in W\}$ then $C'=C \cap W_{A}$ is a geography. 
	\end{lemma}
	\begin{proof}
		Index all of the faces of every polytope in the decomposition by $\mathcal{D}_{i}$ as $F_{j}$. Then for $C'$ to be a geography it is enough to know that intersecting with $W$ preserves the codimension of the $F_{j}$ meeting $W$. For fixed $j$, however the choices of $W$ such that either $W$ does not meet $F_{j}$ or $F'_{j}=F_{j} \cap W_{A} \subseteq C'$ has the same codimension as $F \subseteq C_{A}$ form an open set in the Grassmanian. Since there are finitely many faces the result holds for suitably general choice of $W$.
	\end{proof}

	

	\begin{lemma}
		Suppose $V$ is a coefficient space which spans $NS(X)$. Let $C$ be any polytope contained $\mathcal{RL}_{A}(V)$, then after perturbing the vertices by an arbitrarily small amount $(C,A,V)$ is a geography.
	\end{lemma}
	\begin{proof}
		Since we can perturb the vertices of $C$ we may suppose it is rational and contained in the interior of $\mathcal{RL}_{A}(V)$. Let $W$ be the minimal coefficient space in $V$ with $C \subseteq W_{A}\cap \mathcal{RL}_{A}(V)$. Since $C$ is contained in the interior of $\mathcal{RL}_{A}(V)$, we can pick an rlt polytope $C'$ which spans $NS(X)$ with $W_{A}\cap C'=C$. Then after a small perturbation of the vertices we may suppose that $W_{A}\cap C'$ is a geography, as required.
	\end{proof}

	\begin{lemma}\cite[Lemma 3.6]{hacon2009sarkisov}\label{amp}
		Let $(X,\Delta)$ be an rlt pair. Suppose that $B-\Delta$ is ample and $f$ is an ample model for $K_{X}+B$. Then $f$ is an MMP for $(X,\Delta)$.
	\end{lemma}
\begin{lemma}\label{geo}
	Suppose that $f_{i}: (X,\Delta) \to (Y_{i},\Delta_{i})$ for $i=1,..n$ are a finite collection of $\mathbb{Q}$-factorial Mori Fibre spaces obtained by running an MMP for an rlt pair $(X,\Delta)$ with $X$ smooth. Then there is a geography $(C,A,V)$ for $(X,\Delta)$ of dimension at most $n$ such that every $\mathcal{D}_{f_{i}}$ is a country. 
	
	Moreover if $g_{i}:Y_{i}\to Z_{i}$ are the Mori Fibrations and we write $h_{i}=g_{i}\circ f_{i}$. Then we may choose $C$ such that $\mathcal{D}_{h_{i}}$ are borders of the $\mathcal{D}_{f_{i}}$ and their interiors are connected by a path through the border of $\mathcal{C}$ contained entirely in the interior of $C$.
\end{lemma}

\begin{proof}

	Pick $A'_{i}$ ample on $Z_{i}$ such that $g_{i}^{*}A_{i}-(K_{Y_{i}}+\Delta_{i})$ is ample. 
	
	We may choose $H$ ample on $X$ whose components span $NS(X)$ together with $A$ ample both sufficiently small such that:
	\begin{itemize}
		\item $(X,H+A)$ is klt
		\item the $A_{i} = g_{i}^{*}A'_{i}-(K_{Y_{i}}+\Delta_{i}+f_{i,*}(A+H))$ are ample
		\item $(X,\Delta+A+H)$ is an rlt pair which is not psuedoeffective
		\item each $f_{i}$ is $(K_{X}+\Delta+A+H)$ negative.
	\end{itemize}

	Further, we may pick $A$ such that it avoids the exceptional loci of the $f_{i}$.
	
	By Lemma \ref{bertini} we can take $B_{i} \sim f_{i}^{*}A_{i}$ such that each $(X,\Delta+H+A+B_{i})$ is rlt. Moreover we can choose the $B_{i}$ such that they share no components with $A$ since the augmented base locus of $B_{i}$ is precisely the exceptional locus of $f_{i}$. Thus the $(X,\Delta+B_{i})$ all have witnesses in some $W$ for which $(X,\Delta+H+A+B_{i})$ have witnesses in $W_{A+H}$.
	
	By construction, then, after adding the components of $H$ to $W$ we have $(X,\Delta+B_{i}+H+A) \in \mathcal{RL}_{A}(W)$, a geography. Further the $f_{i}$ are wlc models of the $(X,\Delta+B_{i}+H+A)$ and the $h_{i}$ are the ample models.
	
	Let $C$ be the convex hull of the $\Delta+B_{i}+H+A$ and $\Delta+H+A$. Since the components of $H$ span NS(X) we can find boundaries in $\mathcal{RL}_{A}(W)$ for which the $f_{i}$ is an ample model. We can freely move the boundaries an arbitrarily small amount such that $C$ meets the interior of each of the $\mathcal{D}_{f_{i}}$ and their borders $\mathcal{D}_{h_{i}}$ while ensuring they are sufficiently general that $C$ is a geography. 
	
	By construction, $C_{-\Delta}$ is contained in the ample cone and $\dim C \leq n$. It remains to check that $\mathcal{D}_{h_{i}}$ are borders of the $\mathcal{D}_{f_{i}}$ and their interiors are connected by a path through the border of $\mathcal{E}(C)$ contained entirely in the interior of $C$.
	
	Since $C$ contains a vertex $D \notin \mathcal{E}(C)$ such that $C_{-D}$ is contained in the effective cone, it is enough to check that for each $i$ the interior of $\mathcal{D}_{h_{i}}$ meets the interior of $C$, but this again is ensured by the construction. Thus we may take $E_{i},E_{j}$ in the interiors of $\mathcal{D}_{h_{i}},\mathcal{D}_{h_{j}}$ respectively and both contained in the interior of $C$. Then the simplex formed by $D,E_{i},E_{j}$ meets the boundary of $\mathcal{E}(C)$ along a path connecting $E_{i}$ and $E_{j}$, wholly contained in the interior of $C$.

\end{proof}
%	
%	\begin{lemma}
%		Suppose that $f_{i}: (X,\Delta) \to (Y_{i},\Delta_{i})$ for $i=1,..n$ are a finite collection of $\mathbb{Q}$ factorial minimal models for an rlt pair $(X,\Delta)$ with $X$ smooth. Then there is a geography $(C,A,V)$ for $(X,\Delta)$ such that every $(Y_{i},\Delta_{i})$ is a country. 
%		If $(X,\Delta)$ is pseudo-effective then we can choose such a $C$  with dimension at most $n$.
%		Otherwise we have $g_{i}:(Y_{i},\Delta_{i}) \to Z_{i}$ Mori Fibre spaces. Let $h_{i}:X \to Z$  be the induced map then we may choose $C$ of dimension at most $n+1$ such that each $\mathcal{D}_{h_{i}}(C)$ is a border of $\mathcal{D}_{f_{i}}(C)$.
%	\end{lemma}
%
%	\begin{proof}
%	Suppose first that some $f_{i}$ is a log terminal model, in this case it must be that all the $f_{i}$ are.
%	
%	Thus $f_{i,*}(K_{X}+\Delta)=K_{Y_{i}}+\Delta_{i}$ is nef. Take $A_{i}$ ample on $X_{i}$. Then each $f_{i}^{*}A_{i}$ is big and nef, so we may choose $B_{i} \sim f_{i}^{*}A_{i}$ such that each $(X,\Delta+B_{i})$ is rlt and so too is $(X,\Delta+B)$ is rlt, for $B=\sum B_{i}$.
%	
%	By construction each $f_{i}$ is an ample model for the pair $(X,\Delta+B_{i})$. Let $W$ be a coefficient space such that $(X,\Delta)$ and $(X,\Delta+B_{i})$ are all have witnesses in $W$.
%	
%	Since $X$ is $\mathbb{Q}$-factorial we can take $H$ ample on $X$ whose components generate $NS(X)$. Choose $A$ not containing any component of $W$ and $H$. Shrinking $A+H$ we may suppose that $(X,\Delta+B+H+A)$ is rlt, that every pair $(X,\Delta+B_{i}+H+A)$ is rlt and moreover that each $f_{i}$ is still an ample model of $(X,\Delta+B_{i}+H+A)$. Since we are free to shrink as much as needed, we may suppose that these pairs have witnesses in $W_{A+H}$.
%	
%	After shifting $W$ by $H$ then we have that $(X,\Delta+B_{i}+H+A) \in \mathcal{RL}_{A}(W)$. By construction $(\Delta+B_{i}+H+A)-\Delta$ is ample. Let $C$ be the polytope spanned by the $R(X,\Delta+B_{i}+H+A)$ then if $B \in C$ we have $B-\Delta $ ample. Perturbing $C$ we may suppose that it is a geography while ensuring that $C-\Delta$ is in the ample cone and the ith vertex of $C$, inheriting the ordering from the construction, is contained in $\mathcal{A}_{f_{i}}$. 
%	
%	In particular then each $\mathcal{D}_{f_{i}}$ is a country of $C$, which is necessarily a geography for $(X,\Delta)$.
%	
%	Suppose now that $g_{i}:(Y_{i},\Delta_{i}) \to Z_{i}$ is a Mori Fibre Space, in which case there is such a $g_{j}:Y_{j} \to Z_{j}$ for any other $j$. Pick $A'_{i}$ ample on $Z_{i}$ such that $g_{i}^{*}A_{i}-(K_{Y_{i}}+\Delta_{i})$ is ample. 
%	
%	We may choose $H$ ample on $X$ whose components span $NS(X)$ sufficiently small together with $A$ ample such that $A_{i} = g_{i}^{*}A'_{i}-(K_{Y_{i}}+\Delta_{i}+f_{i,*}(A+H))$ is ample and that $(X,\Delta+A+H)$ is an rlt which is not psuedoeffective. We may pick $A$ such that it avoids the exceptional loci of the $f_{i}$.
%	
%	As before we can take $B_{i} \sim f_{i}^{*}A_{i}$ such that each $(X,\Delta+H+A+B_{i})$ is rlt. Moreover we can choose the $B_{i}$ such that they share no components with $A$. Indeed the augmented base locus of $B_{i}$ is precisely the exceptional locus of $f_{i}$, so by Lemma \ref{bertini} we are free to choose them such that the $(X,\Delta+B_{i})$ all have witnesses in some $W$ for which $(X,\Delta+H+A+B_{i})$ have witnesses in $W_{A+H}$.
%	
%	Thus by construction we have $(X,\Delta+B_{i}+H+A) \in \mathcal{RL}_{A}(W)$, a geography.
%	
%	This time let $C$ be the convex hull of the $\Delta+B_{i}+H+A$ and $\Delta+H+A$. Since the components of $H$ span NS(X) we can find boundaries in $\mathcal{RL}_{A}(W)$ for which the $f_{i}$ is an ample model. We can freely move the boundaries an arbitrarily small amount such that $C$ meets each of the $\mathcal{D}_{f_{i}}$ and their borders $\mathcal{D}_{g_{i}}$ while ensuring they are sufficiently general that $C$ is a geography.
%	
%	By construction, $C-\Delta$ is ample and so $C$ satisfies all the hypotheses. 
%	\end{proof}
	
	
%		\begin{lemma}
%		Let $(C,A,V)$ be a geography on $X$. Take two countries $\mathcal{D}_{f}$ and $\mathcal{D}_{g}$ corresponding to some maps $f:X \dashrightarrow Y$ and $g: X\dashrightarrow Z$. Suppose that $\mathcal{D}_{f}$ is a country and that they meet along a border $\mathcal{B}$ not contained in the boundary of $C$. Suppose further that $\rho(Y) \geq \rho(Z)$. Let $B\in \mathcal{B}$ be an interior point and $\Delta=f_{*}B$
%		
%		Then there is $h: Y \dashrightarrow Z$ a the rational map induced by $\mathcal{B}$. It is either a $(K_{Y}+\Delta)$ flop or a $(K_{Y}+\Delta)$ trivial, divisorial contraction.
%		\end{lemma}
%		\begin{proof}
%		By restricting to a suitably general line segment inside $C$ we may suppose that $C$ is $1$-dimensional with $\mathcal{E}(C)=C$. Then after replacement, $B$ is the unique point of $\mathcal{B}$.
%		
%		Let $\phi: X\to W$ be the ample model for $K_{X}+B$. Then we get induced maps $s:Y \to W$, $t:Z \to W$. Since $B$ is in the interior of a border these maps must have relative Picard number $1$.
%		
%		Suppose that $Z \to W$ has Picard rank $0$. Then $t:Z \to W$ is an isomorphism and we have a $(K_{Y}+\Delta)$ trivial morphism $s: Y \to Z$ of relative Picard rank $1$. Then $s$ cannot be a small contraction since $Z$ is $\mathbb{Q}$-factorial, thus it is a divisorial contraction.
%		
%		Else $\rho(Z/W)=1$, and thus $\rho(Y/W)=1$ also. Now $s,t$ cannot be divisorial else $\phi$ would be a birational morphism of $\mathbb{Q}$ factorial varieties, forcing $\mathcal{B}$ to be a country.
%		
%		Since $B$ is in the interior of $C=\mathcal{E}(C)$, $\phi$ is birational and thus the only possibility which remains is that $s,t$ are small contractions. Since $s$ is $(K_{Y}+\Delta)$ trivial, this is a $K_{Y}+\Delta$ flop.
%		\end{proof}
%		
%		If $(C,A,V)$ is a geography for some pair $(X,B)$ and $f,g$ are ample models as above then, by definition, $(Y,f_{*}B)$ and $(Z,g_{*}B)$ are minimal models for $(X,B)$. 
	
	
	\begin{lemma}\cite[Lemma 3.5]{hacon2009sarkisov}\label{links}
		Let $(C,A,V)$ be a geography on $X$ of dimension $2$. Take two ample classes $\mathcal{D}_{f}$ and $\mathcal{D}_{g}$ corresponding to some maps $f:X \dashrightarrow Y$ and $g: X\dashrightarrow Z$. Suppose that $\mathcal{D}_{f}$ is a country and that they meet along a border $\mathcal{B}$ not contained in the boundary of $C$. Suppose further that $\rho(Y) \geq \rho(Z)$
		
		Let $h: Y \dashrightarrow Z$ be the map induced by $\mathcal{B}$. Take $B$ an interior point of $\mathcal{B}$ and let $\Delta=f_{*}B$, then one of the following holds.
			\begin{enumerate}
			\item $\rho(Y)=\rho(Z)+1$ and $h$ is a $K_{Y}+\Delta$ trivial morphism. Thus either
			\begin{enumerate}[a)]
				\item $h$ is a divisorial contraction and $\mathcal{B} \neq \mathcal{D}_{g}$
				\item $h$ is a small contraction and $\mathcal{B}=\mathcal{D}_{g}$
				\item $h$ is a MFS and $\mathcal{B}=\mathcal{D}_{g}$ is contained in the boundary of $\mathcal{E}(C)$.
			\end{enumerate}
			\item $\rho(W)=\rho(Y)$ and $h$ is a $K_{Y}+\Delta$ flop and $\mathcal{B} \neq \mathcal{D}_{g}$ is not contained in the boundary of $\mathcal{E}(C)$.
		\end{enumerate}
	\end{lemma}
	
	

\section{Sarkisov Program} \label{Sarkisov-section}


Suppose that $f:X \to Z$, $g:Y \to W$ are two Mori Fibre Spaces over $R$. A sarkisov link $s:X \dashrightarrow Y$ is one the following.

\[\begin{tikzcd}
X' \arrow[d] \arrow[r, dotted] \arrow[r, "I", phantom, bend left=49] & Y \arrow[d]  & X' \arrow[r, dotted] \arrow[d] \arrow[r, "II",phantom, bend left=49] & Y' \arrow[d] & X \arrow[r, dotted] \arrow[d] \arrow[r, "III", phantom, bend left=49] & Y' \arrow[d] & X \arrow[d] \arrow[rr, dotted] \arrow[rr, "IV",phantom, bend left] &   & Y \arrow[d]       \\
X \arrow[d]                                                          & W \arrow[ld] & X \arrow[d]                                               & Y \arrow[d]  & Z \arrow[rd]                                              & Y \arrow[d]  & Z \arrow[rd, "p"]                                          &   & W \arrow[ld, "q"'] \\
Z                                                                    &              & Z \arrow[r, equal]                                                         & W            &                                                           & W            &                                                            & T &                  
\end{tikzcd} \]


Such that the following holds:
\begin{itemize}
	\item There is an rlt pair $(X,\Delta)$ or $(X',\Delta')$ as appropriate such that the horizontal map is a sequence of flops for this pair
	\item Every vertical morphism is a contraction
	\item If the target of a vertical morphism is $X$ or $Y$ then it is an extremal divisorial contraction
	\item Either $p,q$ are both Mori Fibre Spaces (this is type $IV_{m}$) or they are both small contractions (type $IV_{s}$)
\end{itemize}



Fix $X$ and a geography $(C,A,V)$ on $X$.

Let $\Delta$ be a point in the boundary of $\mathcal{E}(C)$ but in the interior of $C$. Let $\mathcal{T}_{1}=\mathcal{D}_{f_{1}},..., \mathcal{T}_{k}=\mathcal{D}_{f_{k}}$ be the countries which meet $\Delta$. Let $\mathcal{B}_{i}$ be the borders $\mathcal{T}_{i}$ meeting $B$ such that after reordering we have $\mathcal{B}_{i}=\mathcal{T}_{i}\cap \mathcal{T}_{i+1}$ for $1 \leq i \leq k-1$. Then $\mathcal{B}_{0}, \mathcal{B}_{k}$ are contained in the boundary of $\mathcal{E}(C)$. Let $g_{i}:X \to S_{i}$ be the ample models associated to the interiors of $\mathcal{B}_{i}$

Relabel $\phi=f_{0}:X \dashrightarrow Y$, $S=S_{0}$, $\psi=f_{k}\dashrightarrow W$ and $T=S_{k}$. Then we have $p,q$ with $p \circ \phi=g_{0}$ and $q \circ \psi =g_{k}$.


%	. Let $h:X \dashrightarrow Z$ be the ample model of $\Delta$.

%	We label these as follows
%	
%	\begin{itemize}
%		\item $f_{i}:X \to Y_{i}$
%		\item $\Delta_{i}=f_{i,*}\Delta$
%		\item $Y=Y_{1}$
%		\item $\phi=f_{1}:X \dashrightarrow Y$
%		\item $\Delta_{Y}=\phi_{*}\Delta$
%		\item $Y'=Y_{2}$ if $k \geq 3$
%		\item $p:Y \to S=S_{0}$ the morphism induced by $g_{0}$
%		\item $s:S \to Z$ induced by $h$ 
%		\item $W=Y_{k}$
%		\item $\psi=f_{k}: X\dashrightarrow W$
%		\item $W'=Y_{k-1}$ if $k \geq 3$
%		\item $q:W \to T=S_{k}$ the morphism induced by $g_{k}$.
%		\item $t:T \to Z$ induced by $h$
%		\item $h_{i}:Y_{i} \to Z$ induced by $h$
%	\end{itemize}
%

	\begin{theorem}\cite[Theorem 3.7]{hacon2009sarkisov}\label{islinked}
		With notation as above, suppose $B$ is any divisor with $\Delta-B$ ample. Then $q:Y \to S$ and $q: W \to T$ are two Mori Fibre spaces obtained by running $(X,B)$ MMPs and they are connected by Sarkisov links.
	\end{theorem}
%	
%	\begin{proof}
%		That $\phi$, $\psi$ are obtained by running MMP's follows straight from Lemma \cite{amp}. By Lemma \ref{links}, $p:Y \to S$ and $q:W \to S_{1}$ are both Mori Fibre Spaces.We also have that $h_{i}: Y_{i} \to Z$ have $\rho(Y_{i}/Z) \leq 2$ and they factor as $Y_{i} \to S_{j} \to Z$ for $j=i,i-1$ where $Y_{i} \to S_{j}$ has Picard rank at most $1$. 
%		
%		Thus if $\rho(h_{i})=1$ then either $h_{i}=id:S_{j} \to Z$ and $Y_{i} \to S_{j}$ is a MFS, in which case $S_{j}=S$ or $S_{j}=T$, or $Y_{i} \to S_{j}$ is the identity. Conversely if $Y_{i} \to S_{j}$ is the identity for $j=i$ or $j=i+1$ then $Y_{i} \to Z$ has relative picard rank $1$, and thus $\mathcal{D}_{h} \cap \mathcal{D}_{\phi}$ has dimension $1$. In particular $S_{k}=Z$ for $j \neq k \in \{i,i+1\}$, thus $\rho(Y_{i}/Z)=1$ only when $i=0$ or $i=k$.  
%		
%		Suppose $k=2$. If $Y,W$ both have Picard rank $1$ over $Z$ and the map $Y \to W$ is a flop and $Y \to Z$, $W \to Z$ are both Mori Fibre Spaces. This is type II. If both $Y$ and $W$ have Picard rank $2$ over $Z$ then $Y \to S$, $W \to T$ are MFS. This is type IV. Else we have one of each, say $\rho(W/Z)=2$. Then $W \to Y$ is a divisorial contraction, $Y \to Z$ is a MFS and so is $W \to T$. This is type I. If instead $\rho(Y/Z)=2$ then it is type III.
%		
%		If $k \geq 3$ then  $Y_{i} \dashrightarrow Y_{i-1}$ is a flop. Hence we need only consider the possibilities involving $Y, Y', W,W'$. 
%		
%		If $Y \to Z$ has Picard rank $1$ then $Y' \to Y$ is a divisorial contraction. If it is has rank $2$ then $Y'\dashrightarrow Y$ is a flop. The same holds for $W' \to W$. So as in the $K=2$ case we can classify by the Picard rank of $Y \to Z$ and $W \to Z$.
%		
%		If $\rho(Y/Z)=\rho(W/Z)=1$ then $Z=S=T$ and it is type $II$. If $\rho(Y/Z)= \neq \rho(W/Z)$ then one side is a MFS and the other is not, so we get type I or III. If $\rho(Y/Z)=\rho(W/Z)=2$ then $Y \dashrightarrow W$ is a composition of flops and $\phi, \psi$ are Mori Fibre Spaces to $S,T \neq Z$ this is type IV.
%		
%		We can further describe type IV links. Suppose first for contradiction $s:S \to Z$ is a divisorial contraction. Let $F$ be the contracted divisor and $E$ its preimage on $Y$. Then $p^{*}F=mE$ for some $m \geq 0$ and $K_{Y}+\Delta_{Y}+tE$ is rlt for sufficiently $t >0$. If we run an MMP for this pair over $R$ we just contract $E$ to get $\hat{W}$. Since we may take $t$ to be as small we we'd like, this must be a $K_{Y}+\Delta$ trivial contraction and this $\hat{W}=Y$, so this is type III a contradiction.
%		
%		Equally $t$ cannot be divisorial. And thus $s,t$ are Mori Fibre spaces if and only if $Z$ is $\mathbb{Q}$-factorial. Otherwise they are both small. These correspond to $\text{IV}_{m}$ and $\text{IV}_{s}$ respectively.
%		
%	\end{proof}

	\begin{theorem}\label{sarkisov}
		Let $g_{1}:Y_{1} \to Z_{1}$ and $g_{2}:Y_{2} \to Z_{2}$ be two Sarkisov related Mori fibre spaces. Then they are connected by Sarkisov links.
	\end{theorem}

	\begin{proof}
		
		By assumption these Mori fibre spaces are outputs of an MMP for some pair $(X,\Delta)$. Replacing $X$ with a suitable resolution, we may suppose that $X$ is smooth and admits morphisms $f_{i}:X \to Y_{i}$. Let $h_{i}=g_{i} \circ f_{i}$ then by Lemma $\ref{geo}$ there is a geography for $(X,\Delta)$ of dimension $2$ such that the $\mathcal{D}_{f_{i}}(C)$ are countries and the interiors of the $\mathcal{D}_{h_{i}}$ are connected by a path along the boundary of $\mathcal{E}(C)$. 
		
		Each ridge in this path corresponds to a Sarkisov link by Theorem \ref{islinked}. Thus following the path gives a (non-unique) decomposition of $f_{2} \circ f_{1}^{-1} \colon Y_{1} \dashrightarrow Y_{2}$ into Sarkisov links. Since $\mathcal{E}(C)$ is a rational polytope, there are finitely many links.
		
		
	\end{proof}

%	
%	\bibliography{Refs}
%	\bibliographystyle{alpha}
%\end{document}

