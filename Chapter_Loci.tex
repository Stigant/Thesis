\documentclass[a4paper,12pt]{book}
\usepackage[a4paper, margin=1in]{geometry}
\setlength{\parindent}{0cm}
\setlength{\parskip}{1\baselineskip}

\setcounter{secnumdepth}{4}
\usepackage{amsthm}
\usepackage{aliascnt}
\usepackage{amsmath,amssymb,amsfonts}
\usepackage{hyperref}
\usepackage[shortlabels]{enumitem}


\RequirePackage[dvipsnames,usenames]{xcolor}
\usepackage{mathtools}
%\usepackage{showkeys}
\usepackage[abbrev,alphabetic]{amsrefs}
\usepackage[all]{xy}
\usepackage{tikz}
\usepackage{tikz-cd}
\usepackage{systeme}

\usepackage{tikz-cd}

\newcommand{\basetheorem}[3]{
	\newtheorem{#1}{#2}[#3]
	\newtheorem*{#1*}{#2}
	\expandafter\def\csname #1autorefname\endcsname{#2}
}
\newcommand{\maketheorem}[3]{
	\newaliascnt{#1}{#3}
	\newtheorem{#1}[#1]{#2}
	\aliascntresetthe{#1}
	\expandafter\def\csname #1autorefname\endcsname{#2}
	\newtheorem{#1*}{#2}
}

%\theoremstyle{plain}
\basetheorem{theorem}{Theorem}{section}
\maketheorem{definition}{Definition}{theorem}
\maketheorem{lemma}{Lemma}{theorem}
\maketheorem{example}{Example}{theorem}
\maketheorem{corollary}{Corollary}{theorem}
\maketheorem{conjecture}{Conjecture}{theorem}
\maketheorem{proposition}{Proposition}{theorem}
\maketheorem{question}{Question}{theorem}
\maketheorem{remark}{Remark}{theorem}
%\basetheorem{theorem*}{Theorem}

\basetheorem{theo}{Theorem}{}
\newcounter{tmp}

\setcounter{tmp}{\value{theo}}% store current value of theorem counter
\setcounter{theo}{0} %assign desired value to theorem counter
\renewcommand\thetheo{\Alph{theo}}% locally redefine the representation of the theorem counter


\makeatletter
\newcommand{\newreptheorem}[2]{\newtheorem*{rep@#1}{\rep@title}\newenvironment{rep#1}[1]{\def\rep@title{#2 \ref*{##1}}\begin{rep@#1}}{\end{rep@#1}}}
\makeatother
\makeatletter
\@namedef{subjclassname@2020}{%
	\textup{2020} Mathematics Subject Classification}
\makeatother


%\newreptheorem{theorem}{Theorem}

%begin: Iacopo-defined newcommands
\DeclareMathOperator{\Spec}{Spec}
\DeclareMathOperator{\Proj}{Proj}
\newcommand{\bP}{\mathbb{P}}
\newcommand{\bR}{\mathbb{R}}
\newcommand{\bQ}{\mathbb{Q}}
\newcommand{\bN}{\mathbb{N}}
\newcommand{\bZ}{\mathbb{Z}}
\newcommand{\fB}{\mathbf{B}}
\newcommand{\fM}{\mathbf{M}}
\newcommand{\charac}{\textup{char }}
\newcommand{\id}{\textup{id}}
\newcommand{\Alb}{\textup{Alb}}
\newcommand{\cO}{\mathcal{O}}
\newcommand{\red}{\textup{red}}
\newcommand{\lct}{\textup{lct}}
\newcommand{\exc}{\textup{Ex}}
\newcommand{\coeff}{\textup{coeff}}
\newcommand{\cent}{\textup{centre}}
\newcommand{\codim}{\textup{codim}}
\newcommand{\textoverline}[1]{$\overline{\mbox{#1}}$}
\newcommand{\rk}{\textup{rk}}
\newcommand{\WDiv}{\textup{WDiv}}
%end: Iacopo-defined newcommands


\newcommand{\A}{\mathcal{A}}
\newcommand{\B}{\mathcal{B}}
\newcommand{\C}{\mathcal{C}}
\newcommand{\D}{\Delta}
\newcommand{\Vol}{\text{Vol}}
\newcommand{\Sing}{\text{Sing}}
\newcommand{\E}{\mathcal{E}}
\newcommand{\F}{\mathcal{F}}
\newcommand{\PP}{\mathcal{P}}
\newcommand{\HH}{\mathcal{H}}
\newcommand{\SB}{\mathbf{SB}}
\newcommand{\BB}{\mathbf{B}}
\newcommand{\BS}{\mathbf{B}_{+}}
\newcommand{\disc}{\text{Discrepancy}}


\newcommand{\ta}[1]{\mathcal{A}^{\leq #1}}
\newcommand{\at}[1]{\mathcal{A}^{\geq #1}}
\newcommand{\tb}[1]{\mathcal{B}^{\leq #1}}
\newcommand{\bt}[1]{\mathcal{B}^{\geq #1}}
\newcommand{\tc}[1]{\mathcal{C}^{\leq #1}}
\newcommand{\ct}[1]{\mathcal{C}^{\geq #1}}
\newcommand{\nklt}{\text{Nklt}}
\newcommand{\Ht}[1]{H^{i}_{t}}
\newcommand{\orth}{^{\perp}}
\newcommand{\Hom}{\textup{Hom}}
\newcommand{\perf}{\textup{perf}}
\newcommand{\Fe}{F^{e}_{*}}
\newcommand{\Fn}[1]{F^{#1}_{*}}
\newcommand{\trip}{(R,\Delta, \alpha_{\bullet})}
\newcommand{\ai}{\alpha_{\bullet}}
\newcommand{\im}{\text{Im}}
\newcommand{\ox}{\mathcal{O}_{X}}
\newcommand{\me}{M^{e}_{\Delta,a^{t}}}
\newcommand{\psim}{\sim_{\mathbb{Z}_{(p)}}}
\newcommand{\zp}{\mathbb{Z}_{(p)}}
\newcommand{\Xde}[1]{\mathcal{X}_{\delta,\epsilon,#1}}
\newcommand{\Pde}[1]{\mathcal{P}_{\delta,\epsilon,#1}}
%

\newtheorem{case}{Case}

\usepackage{xcolor}
\newcommand\myworries[1]{\textcolor{red}{#1}}

%\usepackage{subfiles}


\begin{document}

\chapter{The Augmented Base Locus in Mixed Characteristic}

This chapter studies the stable and augmented base loci of nef divisors in mixed characteristic. Generally under the further assumption that the divisor is semiample in characteristic $0$.

We give a characterisation of the augmented base locus in this setting.

\begin{theorem}[\autoref{Main_Loci}]
Let $X$ be a projective scheme over an excellent Noetherian base $S$ with $L$ a nef line bundle on $X$. 
Suppose that one of the following holds:
\begin{enumerate}
	\item $S_{\mathbb{Q}}$ has dimension $0$;
	\item $L|_{X_{\mathbb{Q}}}$ is semiample;
\end{enumerate}

Then $\BS(L)=\mathbb{E}(L)$.
\end{theorem}

We also extend the semiampleness result of \cite{witaszek2020keel} to show that there is an equality of stable base loci when the characteristic $0$ part is semiample.
 
 \begin{theorem}[\autoref{Main_Loci2}]
 	Suppose that $X$ is a projective scheme over an excellent Noetherian base with $L$ a nef line bundle on $X$. Then $\SB(L)=\SB(L|_{\mathbb{E}(L)})$ so long as $L|_{X_{\mathbb{Q}}}$ is semiample.
 \end{theorem}


\section{Preliminaries}

We will work exclusively with line bundles. Since the schemes we work with need not be normal, line bundles are not the same as Cartier divisors, however we typically use the traditional notation for divisors as we still sometimes treat line bundles as Cartier divisors when appropriate. That is we write the tensor product of $L,L'$ as $L+L'$, $L^{\otimes k}$ is often written $kL$ and given $f:Y \to X$, then $f^{*}L=L|_{Y}$ is often written $\ox[Y](L)$, including for $Y=X, f=id$. 

Since the questions considered are local on the base, it suffices to work only with affine bases. In particular, for notational simplicity, $H^{i}(X,L)$ will often be used to denote the higher derived pushforwards of $L$ by $X \to S$. 
	
\begin{definition}
	Let $L$ be a line bundle on a projective Noetherian scheme $X$ over some Noetherian scheme $S$. Then base locus is given as 
	$$\BB(L)= \bigcap_{s \in H^{0}(X,L)} Z(s)_{red}$$
	where $Z(s)$ is the zero set of $s$ equipped with the obvious scheme structure. The stable base locus is then
	$$\SB(L)=\bigcap_{m \geq 0}\BB(mL).$$
	Fix an ample line bundle $A$. The augmented base locus is given as 
	$$\BS(L)=\bigcap_{m \geq 0}\SB(mL-A)$$
	and is independent of the choice of $A$.
	
\end{definition}

We could also write \[\BS(L)= \bigcap_{A \text{ ample, } m \geq 0}\SB(mL-A)\] for a definition that involves no choice of ample line bundle. By Noetherianity if we choose $m$ sufficiently large and divisible then in fact $\BS(L)=\SB(mL-A)$.

 \begin{definition}
	Let $L$ be a line bundle on a projective scheme $X$. The exceptional locus, $\mathbb{E}(L)$, is the union of integral subschemes on which $L$ is not big.
\end{definition}

The previous two definitions are invariant under scaling by $n \in \mathbb{N}_{\geq 0}$ and line bundles will frequently be replaced with higher multiples.

\begin{theorem}\cite{witaszek2020keel}[Theorem 1.10]\label{red}
	Suppose that $X$ is a projective scheme over an excellent Noetherian base $S$ and $L$ is a nef line bundle on $X$. Then if $L|_{X_{red}}$ and $L|_{X_{\mathbb{Q}}}$ are semiample so too is $L$.
\end{theorem}


\begin{theorem}\cite{keeler2003ample}[Theorem 1.5]\label{{Keeler}}
	Let $X$ be a projective scheme over a Noetherian ring, $\A$ an ample line bundle and $\F$ a coherent sheaf. Then there is some $m_{0}$ with 
	
	\[H^{i}(X, \F \otimes \A^{m} \otimes \mathcal{N})=0\]
	for all $i>0, m \geq m_{0}$ and all nef line bundles $\mathcal{N}$.
\end{theorem}


\begin{lemma}\cite{cascini2014augmented}[Lemma 2.2]\label{sec-growth}
	Let $X$ be an n-dimensional projective scheme over a field $k$ and $L$ a line bundle on $X$.
	For every coherent sheaf $\F$ on $X$, there is $C > 0$ such that $h^{0}(X, \F\otimes L^{m}) \leq C m^{n}$ for every $m \geq 1$.
\end{lemma}

\begin{lemma}\label{bigsecs}
	Let $X$ be a reduced projective scheme over a ring $R$ and $L,A$ line bundles on $X$ with $A$ ample. Then for large $m$ and general $s \in H^{0}(X,mL-A)$ and any irreducible component $Y$ of $X$ with $L|_{Y}$ big we have $Y \not \subseteq Z(s)$.
\end{lemma}

\begin{proof}
	
	Let $f:X \to S$ be the structure morphism. Suppose for contradiction that $f_{*}\ox(mL-A) \to f_{*}\ox[Y](mL-A)$ is the zero map for infinitely many $m$.
	
	Let $W$ be the union of the other components of $X$ so that we have a short exact sequence 
	
	\[0 \to \ox \to \ox[Y]\oplus \ox[W] \to \ox[Y\cap W] \to 0\]
	
	where $Y,W$ are given the reduced subscheme structure. For convenience we write $Z=Y\cap W$
	
	Tensoring and pushing forwards we get 
	\[0 \to f_{*}\ox(mL-A)\to f_{*}\ox[Y](mL-A)\oplus f_{*}\ox[W](mL-A) \to f_{*}\ox[Z](mL-A) \]
	
	In particular if $f_{*}\ox(mL-A) \to f_{*}\ox[Y](mL-A)$ is the zero map, we must have an injection $ f_{*}\ox[Y](mL-A) \hookrightarrow f_{*}\ox[Z](mL-A)$. Let $V=f(Y)$ and $g=f|_{Y}:Y \to V$. Then we may view $\ox[Y](mL-A), \ox[Z](mL-A)$ as sheaves on $Y$, then there is a corresponding injection $g_{*}\ox[Y](mL-A) \hookrightarrow g_{*}\ox[Z](mL-A)$ since the pushforward is left exact. Since $Y$ is irreducible so too is $V$ and hence we may pull back to the generic point $\nu$ of $V$.
	
	Now we have that $Y_{\nu}$ is a projective scheme over $K(V)$ of dimension say $n$. Equally $Z_{\nu}$ is a closed subscheme of $Y_{\nu}$ of dimension at most $n-1$. We now find a contradiction by counting sections over $K(V)$.
	
	On the one hand we have an injection $$H^{0}(Y_{\nu},\ox[Y_{\nu}](mL-A)) \hookrightarrow H^{0}(Z_{\nu}, \ox[Z_{\nu}](mL-A)),$$ which ensures that there is $C > 0$ such that $h^{0}(Y_{\nu}, \ox[Y_{\nu}](mL-A)) \leq C m^{n-1}$ for every $m \geq 1$ by \autoref{sec-growth}. On the other, $kL|_{Y_{\nu}}$ is big, and $Y_{\nu}$ is integral, thus $h^{0}(Y_{\nu}, \ox[Y_{\nu}](mL-A))$ grows like $m^{n}$ by \cite[Lemma 4.2]{birkar2017augmented}. This is a contradiction and the result follows.		
\end{proof}

\begin{remark}\label{powers}
	
	When $X$ is a reduced scheme and $X=X_{1} \cup X_{2}$ (as topological spaces) for closed subschemes $X_{1},X_{2}$ we have a short exact sequence 
	\[0 \to \ox \to \ox[X_{1}]\oplus \ox[X_{2}] \to \ox[X_{1}\cap X_{2}] \to 0\]
	as used above. In particular if $L$ is a line bundle on $X$ with sections $s_{1},s_{2}$ on $X_{1},X_{2}$ respectively which agree on $X_{1}\cap X_{2}$ then they glue to a section of $L$ on $X$.
	
	This is not the case when $X$ is reducible. If $X_{j}$ are given by ideal schemes $I_{j}$ then it need not be the case that $I_{1} \cap I_{2} = 0$. However replacing $I_{1}$ with a higher power we may suppose that this is the case, see for instance \cite[Tag 01YC]{stacks-project}. In particular we may always choose subscheme structures such that the short exact sequence
	\[0 \to \ox \to \ox[X_{1}]\oplus \ox[X_{2}] \to \ox[X_{1}\cap X_{2}] \to 0\]
	still holds. When we work with components of a reducible scheme we can always chose the subscheme structure in this fashion, and in particular we will always be able to glue appropriate sections.
	
\end{remark}

\begin{lemma}\label{Blowup-close}\cite{eisenbud2000geometry}[Proposition IV-21]
	Let $X$ be a scheme and $Z \subseteq X$ a subscheme with $Y \to X$ the blowup of $X$ along $Z$. If $f:X' \to X$ is any morphism and we write $Z'=f^{-1}Z$, then the closure $W$ of $\pi_{X'}^{-1}(X'\setminus Z')$ inside $X' \times_{X} Y$ is exactly the blowup of $X'$ along $Z'$.
\end{lemma}

\begin{lemma}\label{Blowup-red}\cite[Tag 0808]{stacks-project}
	Let $X$ be a scheme. Let $I\subseteq \ox$ be a quasi-coherent sheaf of ideals. If $X$ is reduced, then the blowup $X'$ of $X$ along $I$ is reduced.
	
\end{lemma}

Together these tell us that 'the blowup of the reduction is the reduction of the blowup'. More precisely we have the following.

\begin{lemma}\label{Blowup}
	
	Let $X$ be a scheme and $Z$ a proper closed subscheme of $X_{red}$. Let $\pi:X' \to X$ be the blowup of $X$ along $Z$, viewed as a subscheme of $X$. Let $Y$ be the blowup of $X_{red}$ along $Z$, then we have isomorphisms
	
	\[Y \simeq X'\times_{X} X_{red} \simeq X'_{red}\]
	
\end{lemma}
\begin{proof}
	
	First we observe that $X'\times_{X} X_{red} \simeq X'_{red}$. Indeed if $f:Z \to X'$ is a morphism from a reduced scheme, then we have a composition $g=\pi \circ f: Z \to X$. And thus a unique induced morphism $Z \to X_{red}$. By definition this induces a unique morphism $Z \to X'\times_{X} X_{red}$ and hence $X'\times_{X} X_{red}$ satisfies the universal property of the reduced subscheme, ensuring that $X'\times_{X} X_{red} \simeq X'_{red}$.
	
	Now by \autoref{Blowup} we have that $Y$ is the closure of $(X_{red}\setminus Z)$ inside $X'\times_{X} X_{red}$. However $X_{red}\setminus Z$ is a dense subscheme and so $Y$ is precisely the reduced subscheme of $X'\times_{X} X_{red}$, but then in fact they are equal as $X'\times_{X} X_{red}$ is already reduced.
	
	
\end{proof}

\begin{lemma}[Elimination of Indeterminacy by blowups]\label{elim}
	Let $f: X \dashrightarrow Y$ be a rational map of $S$ schemes associated to an $S$-linear system $|V|\subseteq H^{0}(X,L)$ without fixed part, then there is $Z$ with maps $\phi_{1}:Z \to X$, $\phi_{2}:Z \to Y$ such that $\phi_{1}^{*}L=M+F$ for $M$ a line bundle globally generated by $\phi_{1}^{*}|V|$. Here $F \geq 0$ is such that $\ox[Y](-F)$ is a line bundle, $\phi_{1}(F)=\BB|V|$ as reduced schemes and $\phi_{2}=f\circ \phi_{1}$. Further we may construct $Z \to X$ as a blowup of $X$.
\end{lemma}

\begin{proof}
	
	Consider the following morphism of line bundles
	$V \otimes L^{-1} \to \ox$
	and let $\mathcal{I}$ be the image. Then $\mathcal{I} \otimes L$ is the image of $V \otimes \ox \to L$, in particular the support of $\mathcal{I}$ is exactly $\BB|V|$.
	
	Let $\pi:Z \to X$ be the blowup of $X$ along $\mathcal{I}$. We then have $\pi^{-1}\mathcal{I}\cdot \ox[Z]=\ox[Z](-F)$ for some $F$ an effective Cartier divisor. Hence we have
	\[\pi^{*}(V\otimes L) \twoheadrightarrow \ox[Z](-F) \hookrightarrow \ox[Z] \]
	where the first map is surjective by right exactness of the pullback functor. Tensoring by $\pi^{*}L$ then gives the following.
	
	\[\pi^{*}(V\otimes \ox[z]) \twoheadrightarrow \pi^{*}L(-F) \hookrightarrow \pi^{*}L \]
	
	In particular the line bundle in the middle, which we may write $M$ is globally generated by sections indexed by $\pi^{*}|V|$ and we have $M=\pi^{*}L(-F)$ by construction. Clearly $\pi(F)$ is the support of $\mathcal{I}$, which is nothing but $\BB|V|$. Since $M$ is globally generated it defines a morphism $\phi_{2}:=\phi_{\pi^{*}|V|}:Z \to Y$ and as $\phi_{1}:=\pi$ is an isomorphism away from $F$ the sections in $\pi^{*}|V|$ agree with those of $|V|$ on this locus. Hence $\phi_{\pi^{*}|V|}$ agrees with $f$ here, that is $\phi_{2}=f\circ \phi_{1}$ as required. 
\end{proof}

%\begin{lemma}[Elimination of Indeterminacy by blowups]\label{elim}
%	
%	Let $(X,L,V)$ be a linear system. Then there is a blowup $\pi:Z \to X$ and a globally generated line bundle $M$ such that $\pi: Z\to X$ induces a morphism of linear systems. Further $\pi$ is an isomorphism of linear systems away from $\BB(V)$. In particular if $F=\pi^{*}L-M$ is the divisor corresponding to the injection $M \to \pi^{*}L$ then $\pi(F)=\BB(V)$ as reduced schemes.
%	
%\end{lemma}
%
%\begin{proof}
%	
%	Let $\mathcal{I}$ be the image of $V \otimes L^{-1} \to \ox$.
%	
%	Let $\pi:Z \to X$ be the blowup of $X$ along $\mathcal{I}$. We then have $\pi^{-1}\mathcal{I}\cdot \ox[Z]=\ox[Z](-F)$ for some $F$ an effective Cartier divisor. Hence we have
%	\[\pi^{*}(V\otimes L) \twoheadrightarrow \ox[Z](-F) \hookrightarrow \ox[Z] \]
%	where the first map is surjective by right exactness of the pullback functor.
%	
%	Tensoring by $\pi^{*}L$ then gives the following.
%	
%	\[\pi^{*}(V\otimes \ox[z]) \twoheadrightarrow \pi^{*}L(-F) \hookrightarrow \pi^{*}L \]
%	
%	In particular the line bundle in the middle, which we may write $M$ is globally generated by sections indexed by $\pi^{*}V$ and we have $M=\pi^{*}L(-F)$ by construction. Clearly $\pi(F)$ is the co-support of $\mathcal{I}$, which is nothing but $\BB(V)$. Hence if we let $W$ be the sections of $M$ corresponding to $\pi^{*}V$ then $(Z,M,W)$ admits a surjective morphism of linear systems to $(X,L,V)$.
%\end{proof}

\begin{lemma}\label{ampall}
	
	Let $H$ be a very ample divisor on $X$. Suppose that $s_{i}$ are sections of $H$ which induce a closed immersion $X \to \mathbb{P}^{V}$. Let $V$ be the submodule generated by the $s_{i}$.
	
	Then for $k$ sufficiently large we have that $V^{\otimes k}=H^{0}(X,H^{k})$.
	
\end{lemma}

\begin{proof}
	
	Thought of as a subscheme of $\mathbb{P}^{V}$, $X$ is cut out by an ideal sheaf $\mathcal{I}$. Hence we have 
	
	\[0 \to \mathcal{I}\otimes \ox[\mathbb{P}^{V}](k) \to \ox[\mathbb{P}^{V}](k) \to H^{k} \to 0.\]
	
	Since $H^{1}(\mathbb{P}^{V}, \mathcal{I}\otimes \ox[\mathbb{P}^{V}](k))=0$ for large $k$, we get a surjection $$H^{0}(\mathbb{P}^{V},\ox[\mathbb{P}^{V}](k)) \to H^{0}(X,H^{k}).$$ However, the image of this map is precisely $V^{\otimes k}$ since we have $H^{0}(\mathbb{P}^{V},\ox[\mathbb{P}^{V}](k))=\bigotimes_{1}^{k}H^{0}(\mathbb{P}^{V},\ox[\mathbb{P}^{V}](1))$.
	
	
	
\end{proof}

\section{Stable Base Loci}

In this section we will examine the stable base locus of line bundles which are semiample over $\mathbb{Q}$. This is then applied to the case of a big and nef line bundle restricted to its exceptional locus.  We begin with an extension of \cite[Theorem 1.10]{witaszek2020keel}. The proof follows the same structure, however more care is needed to keep track of sections.\\

If $L$ is a line bundle on $X$, semiample over $\mathbb{Q}$, we would like to claim that $\SB(L)=\SB(L|_{X_{red}})$. If $L$ or $L|_{X_{red}}$ is semiample then this follows from \cite[Thereom 1.10]{witaszek2020keel}. We would then like to prove the general case by blowing up the base locus of $L|_{X_{red}}$ and reducing to the case that the line bundle is semiample on the reduction. Unfortunately if $Y \to X$ is a blowup then the pullback map $H^{0}(X,L) \to H^{0}(Y,\pi^{*}L)$ is, in general, neither injective nor surjective if $X$ is not integral. It is the lack of surjectivity that causes the issues, since we ultimately wish to show the existence of sections on the original scheme.\\ 

Suppose for example $X$ is the union of two normal projective schemes $X_{1}$, $X_{2}$. Then if $\pi:Y \to X$ is the blowup of $X_{2}$, the map factors through the closed immersion $X_{1} \hookrightarrow X$. Of course if $L$ is a line bundle on $X$ then $H^{0}(X,L) \to H^{0}(X_{1},L|_{X_{1}})\simeq H^{0}(Y,\pi^{*}L)$ is typically not a surjection.\\

 The idea in \cite[Thereom 1.10]{witaszek2020keel} is essentially to show that $L$ is semiample by producing a candidate morphism via pushout. Then one can lift sections back to $L$ by building them from suitable sections of $L|_{X_{red}}$ and $L|_{X_{\mathbb{Q}}}$, up to perhaps replacing the line bundle with a higher power. The key remedy then, is to show that if we blow up the base locus of $L|_{X_{red}}$ via $\pi:Y \to X$, we may build sections of $\pi^{*}L$ on $Y$ using only those coming from $X_{red}$ and $X_{\mathbb{Q}}$.\\

\begin{theorem}\label{BaseRed}
		Let $S$ be an excellent, Noetherian scheme, take $X$ a projective scheme over $S$ and $L$ a line bundle on $X$. Write $i:X_{red} \to X$ for the inclusion of the reduced scheme. Suppose that $L_{X_{\mathbb{Q}}}$ is semiample. Then $\SB(L)=\SB(L_{X_{red}})$.
\end{theorem}

\begin{proof}
	
	We always have $\SB(L_{X_{red}}) \subseteq \SB(L)$ since we can pull back sections of $L$, so it suffices to show the converse. We may also freely localise on $S$ and assume that it is an affine, Noetherian $\mathbb{Z}_{(p)}$ scheme. After replacing $L$ with a sufficiently high mulitple, we assume that $\SB(L)=\BB(L)$, $\SB(L_{X_{red}})=\BB(L_{X_{red}})$ and $\SB(L_{X_{\mathbb{Q}}})=\BB(L_{X_{\mathbb{Q}}})$ as reduced schemes.\\
	\\
	\textbf{Step 1: Blow-up the base locus.}\\
	
	Fix a generating set $s_{i}$ of $H^{0}(X,L_{X_{red}})$.	By \autoref{elim} the blowup $Y_{red} \to X_{red}$ along $Z=\BB(V_{red})$ eliminates the indeterminacy of $L_{red}$. Let $Y \to X$ be the blowup along $X$, viewed here a subscheme of $Z$. Then the reduction of $Y$ is $Y_{red}$ by \autoref{Blowup}.
	
	Let $F$ be the exceptional divisor so that $M = \pi^{*}L(-F)$ is basepoint free. Note that since $L$ is semiample on $X_{\mathbb{Q}}$, we have that $M|_{Y_{\mathbb{Q}}}=\pi^{*}L|_{Y_{\mathbb{Q}}}$. We fix a generating set $t_{i}$ of of $H^{0}(X,M|_{Y_{\mathbb{Q}}})$, which induces a morphism $\phi_{\mathbb{Q}}\colon Y_{\mathbb{Q}} \to Z'_{\mathbb{Q}}$. 
	
	By definition the basis $s_{i}$ of $H^{0}(X,L)$ now induces $\hat{s}_{i}$ in $H^{0}(Y_{red},M|_{Y_{red}})$ which globally generate the line bundle. This induces a morphism $\psi: Y \to Z$ over $S$. Then $Z_{\mathbb{Q}} \to Z'_{\mathbb{Q}}$ is a finite universal homeomorphism by \cite[Tag 02OG]{stacks-project}. 
	
	Now by \cite[Theorem 1.7, Corollary 4.20 and Lemma 2.20]{witaszek2020keel}, there is a scheme $Z'$ and a line bundle $H'$ on $Z$ such that the following diagram commutes at the level of line bundles.
	
	\[\begin{tikzcd}
	{(Y,M)}                                 & {(Y_{\mathbb{Q}},M_{\mathbb{Q}})} \arrow[ddd, bend left=70] \arrow[l]     \\
	{(Y_{red},M_{red})} \arrow[d] \arrow[u] & {(Y_{red, \mathbb{Q}},M_{red, \mathbb{Q}})} \arrow[d] \arrow[u] \arrow[l] \\
	{(Z,H)} \arrow[d]                       & {(Z_{\mathbb{Q}},H_{\mathbb{Q}})} \arrow[l] \arrow[d]                    \\
	{(Z',H')}                               & {(Z'_{\mathbb{Q}},H'_{\mathbb{Q}})} \arrow[l]                            
	\end{tikzcd}\]\\
	\\
	\textbf{Step 2: Find compatible sections.}\\
	
	By definition we have $v_{i} \in H^{0}(Z,H)$ pulling back to the $\hat{s}_{i}$ and $w_{i} \in H^{0}(Z'_{\mathbb{Q}},H'_{\mathbb{Q}})$ pulling back to the $t_{i}$. These sections define closed immersion of $Z, Z'_{\mathbb{Q}}$ into projective space. Let $V,W$ be the submodules generated by the $v_{i}$ and $w_{i}$ respectively.  Then for some $n>>0$ we have that $H'^{\otimes n}$ is very ample, that $V^{\otimes n}=H^{0}(Z,H^{\otimes n})$ and $W^{\otimes n}=H^{0}(Z'_{\mathbb{Q}},H'^{\otimes n}_{\mathbb{Q}})$. The latter two following from \autoref{ampall}.
	
	Hence, after replacing $L$ with a higher power, we can fix generating sets as follows:
	
	\begin{enumerate}
		\item We select a generating set $u_{i}$ for $H^{0}(Z',H')$.
		\item We let $v_{i}=u_{i}|_{Z}$ and $w_{i}=u_{i}|_{Z_{\mathbb{Q}}}$
		\item We reselect $\hat{s}_{i}$, $t_{i}$ to be pullbacks of $v_{i}$ and $w_{i}$
		\item We reselect $s_{i}$ so that they pullback to $\hat{s}_{i}$
	\end{enumerate}
	We are always able to perform selections $3$ and $4$ since the tensor powers of original $v_{i},w_{i}$ generate $H^{0}(Z,H^{\otimes n})$ and $H^{0}(Z'_{\mathbb{Q}},H'^{\otimes n}_{\mathbb{Q}})$. In particular the pullbacks of the $v_{i}$ are contained in the image of $H^{0}(X,L|_{X_{red}})$. By construction, the $v_{i}, w_{i}$ still form a base point free linear system, and thus so do the $\hat{s_{i}}$ and $t_{i}$. Note that we may allow duplicates if needed, e.g. we have $t_{i}=t_{j}$ for $i \neq j$ with no issue.
	
	By construction, we must also have that $\hat{s}_{i}|_{Y_{red,\mathbb{Q}}}=t_{i}|_{Y_{red,\mathbb{Q}}}=\psi^{*}(u_{i}|_{Z_{\mathbb{Q}}})$. Since $\pi: Y \to X$ is an isomorphism over $\mathbb{Q}$ we therefore have that $s_{i}|_{X_{red,\mathbb{Q}}}=t_{i}|_{X_{red,\mathbb{Q}}}$. Since the $\hat{s_{i}}$ globally generate $M$ on $Y$, it must be that $L|_{X_[red}$ is globally generated by the $s_{i}$ away from the exceptional locus of $\pi$, which is precisely the stable base locus of $L$.\\
	\\
	\textbf{Step 3: Glue sections on the original scheme.}\\

	By \cite[Proposition 3.5]{witaszek2020keel}, we have the following commutative diagram. 
	
	\[
	\begin{tikzcd}
	{H^{0}(X,L)^{\perf}} \arrow[d] \arrow[r]                 & {H^{0}(X_{\mathbb{Q}},L_{X_{\mathbb{Q}}})^{\perf}} \arrow[d] \\
	{H^{0}(X_{red},L_{X_{red}})^{\perf}} \arrow[r] \arrow[r] & {H^{0}(X_{red,\mathbb{Q}},L_{X_{red,\mathbb{Q}}})^{\perf}}  
	\end{tikzcd}	
	\]
	Hence we can again replace $L$ with a higher power, and $s_{i}$, $t_{i}$ with the corresponding multiples, such that there are $r_{i} \in H^{0}(X,L)$ with $r_{i}|_{X_{red}}=s_{i}$ and $r_{i}|_{X_{\mathbb{Q}}}=t_{i}$. Once again then $L$ is globally generated by the $r_{i}$ away from $\SB(L|_{X_{red}})$. Hence we have that $\SB(L)\subseteq \SB(L|_{X_{red}})$ as claimed.
\end{proof}

\begin{remark}
	
	In principle the condition that $L|_{X_{\mathbb{Q}}}$ is semiample is not completely necessary. The blowup of $\BB(L|_{X_{red}})$, $\pi: Y \to X$ induces an injection $$H^{0}(X|_{red,\mathbb{Q}},L_{X|_{red,\mathbb{Q}}}) \to H^{0}(Y|_{red,\mathbb{Q}},L_{Y|_{red,\mathbb{Q}}})$$ which is sufficient to allow us to glue sections on the base. Much more care must be taken when replacing $L$ with a higher power in this case, however. 
		
	This would extend the result to the case that $L|_{X_{\mathbb{Q}}}$ becomes basepoint free after we blowup the base locus of $L|_{X_{red}}$. However, it is not clear how this condition could be verified in practice.
	
\end{remark}

We now consider the stable base locus of a big and nef line bundle on restriction to its exceptional locus, under the assumption that the characteristic $0$ part of the line bundle is semiample.

\begin{lemma}
	Let $L$ be a nef line bundle on $X$ projective over an excellent Noetherian base $S$ with and $D$ an effective Cartier divisor such that $L(-D)$ is an ample line bundle. If $L|_{D_{\mathbb{Q}}}$ is semiample then \[\SB(L)=\SB(L|_{D}).\]
\end{lemma}	

\begin{proof}
	Clearly $\SB(L) \subseteq D$ as $L$ is ample away from $D$ and we have $\SB(L|_{D}) \subseteq \SB(L)$ by restriction. Consider the following short exact sequence.
	
	\[0 \to \ox (kL-mD) \to \ox(kL) \to \ox[mD](kL) \to 0\]
	
	By \autoref{{Keeler}}, we may choose $m >>0$ such that $$H^{1}(\ox,kL-mD=mA+(k-m)L)=0$$ for $k \geq m$. Then by \autoref{BaseRed} and the semiampleness assumption, we have $\SB(L|_{D})=\SB(L|_{mD})$ and may pick $k>> m$ with $\SB(L|_{D})=B(kL|_{mD})$ as reduced subschemes of $X$. In particular if $P$ is any closed point of $D$, we may find a section of $kL|_{mD}$ avoiding it, and then lift this to a section of $kL$. Thus $\SB(L)\cap D \subseteq \SB(L|_{D}) $ and the result follows.

\end{proof}


\begin{lemma}
	Suppose that $X$ is a reduced projective scheme over an excellent Noetherian base. Suppose that $L,A$ are line bundles with $L$ nef and $A$ ample. Take $Z=Z(s)$ for some section $s$ of $L-A$. If $L|_{D_{\mathbb{Q}}}$ is semiample then $\SB(L)=\SB(L|_{Z})$.
\end{lemma}

\begin{proof}
	Let $Y_{1}$ be the union of components of $X$ contained in $Z$ and $Y_{2}$ the union of those not contained in $Z$. If either are empty the result is clear so suppose otherwise. As in \autoref{powers}, we give them a subscheme structure and replace $L,A,s$ with higher powers to ensure we may glue appropriate sections.
	
	Let $D=Z \cap Y_{2}$ and $L_{2}=L|_{Y_{2}}$. By assumption $D$ is a Cartier divisor on $Y_{2}$ with $D=(L-A)|_{Y_{2}}$. As above we have 
	\[0 \to \ox[Y_{2}] (kL_{2}-mD) \to \ox[Y_{2}](kL_{2}) \to \ox[mD](kL_{2}) \to 0\]
	and choosing $k > m >>0$ this allows us to lift sections from $kL_{2}|_{mD}$. We then have $\BB(kL|_{mZ})=\SB(L|_{mZ})=\SB(L|_{Z})= \BB(kL|_{Z})$ for large enough $k$ by \autoref{BaseRed}. Now, given any section $t$ of $kL|_{mZ}$ we may restrict it to $D$ and then lift it to $t'$ a section of $kL_{2}$. By construction $t'$ agrees with $t$ on $D=Z \cap Y_{2}$, and since $Y_{1} \subseteq Z$ it follows we may glue $t|_{Y_{1}}$ and $t'$. In particular then we must have $\SB(L)\cap Z = \SB(L|_{Z})$, but since $L$ is ample away from $Z$ the result follows.
\end{proof}


\begin{corollary}\label{Main_Loci2}
	Suppose that $X$ is a projective scheme over an excellent Noetherian base with $L$ a nef line bundle on $X$. Then $\SB(L)=\SB(L|_{\mathbb{E}(L)})$ so long as $L|_{X_{\mathbb{Q}}}$ is semiample.
\end{corollary}
%\myworries{Can we drop the nefness assumption somehow? If $L$ is semiample on E(L) then $L$ is nef, since every curve with $L.C \\
%=0$ must be contained in E(L), hence in fact in $\SB(L|_{\mathbb{E}(L)})$}
\begin{proof}
	By Noetherian induction we may suppose that this holds on every proper closed subscheme. By \autoref{BaseRed} we may suppose that $X$ is reduced and then we may also assume $\mathbb{E}(L) \neq X$, else the result is trivial. Let $X'$ be the union of components on which $L$ is big and $X''$ the union of those on which it is not.
	
	Let $A$ be an ample line bundle and $s$ a general section of $mL-A$, then $Z=Z(s)$ must contain $\mathbb{E}(L)$. By \autoref{bigsecs} we have that $Z \neq X$, since $s$ does not vanish on any component of $X'$. Since $\mathbb{E}(L|_{Z}) = \mathbb{E}(L)\cap Z= \mathbb{E}(L)$ we must have $\SB(L)=\SB(L|_{Z})=\SB(L|_{\mathbb{E}(L)})$ by the induction hypothesis. 
\end{proof}



\section{Augmented Base Loci}

This section considers the augmented base locus of a nef line bundle and its relation to the exceptional locus. This is done largely under the assumption that they are equal in characteristic $0$, before showing this assumption is satisfied in two key cases.


\begin{lemma}\label{amp}
	Let $X$ be a projective scheme, $L$ a line bundle and $A$ a very ample line bundle. Then for $m>>0$ large and divisible we have that 
	
	$$\BS(L)=\BB(mL-A).$$
\end{lemma}

\begin{proof}
	Certainly we have $n$ such that $\BS(L)=\SB(nL-A)$ and thus also $\BS(L)=\BB(nkL-kA)$ for large divisible $k$. Conversely however $\BB(nkL-A)\subseteq \BB(nkL-kA)$ as $A$ is very ample. Since $\BS(L)\subseteq \BB(nkL-A)$ by definition, taking $m=kn$ suffices.
\end{proof}

\begin{lemma}
	Let $X$ be a projective scheme over an excellent Noetherian base with $L$ a nef line bundle on $X$. If $D$ is an effective Cartier divisor with $L(-D)$ an ample line bundle and $\BS(L|_{kD})=\BS(L|_{D})$ for all $k > 0$ then $\BS(L)=\BS(L|_{D})$.
\end{lemma}

\begin{proof}
	Since $D=L-A$ we must have that $\BS(L) \subseteq D$, and conversely $\BS(L|_{D}) \subseteq \BS(L)$ since we may always pullback sections. It suffices to show then that $\BS(L) \subseteq \BS(L|_{D})$ and we need only check this on points inside $D$.
	
	By taking multiples we may freely assume $L-D=2A$ for $A$ very ample. Consider the short exact sequence
	\[0 \to \ox(k(mL-D-A))\to \ox(kmL-kA) \to \ox[kD](mkL-kA) \to 0.\]
	We have that $H^{1}(X,kmL-kD-kA)=H^{1}(X,(k-1)mL+kA)=0$ for $k >>0$ which we now fix and for all $m >0$.
	
	In particular we may lift sections from $\ox[kD](mkL-kA)$ for any $m>0$. By assumption we have $\BS(L|_{kD})=\BS(L|_{D})$ and so we have that $\BS(L|_{kD})=\BB((mkL-kA)_{kD})$ for sufficiently large and divisible $m$. Given this choice we may lift sections avoiding $\BB((mkL-kA)_{kD})$ and thus $\BS(L) \subseteq \BS(L|_{D})$.
\end{proof}

\begin{lemma}\label{reduce}
	Let $X$ be a projective scheme over an excellent Noetherian base with $L$ a nef line bundle on $X$ and $A$ an ample line bundle. If $Z=Z(s)$ for some $s$ a section of $mL-A$ and $\BS(L|_{kZ})=\BS(L|_{Z})$ for all $k \geq 0$ then $\BS(L)=\BS(L|_{Z})$.
\end{lemma}

\begin{proof}
	As above we need only prove that $\BS(L)\subseteq \BS(L|_{Z})$.	Let $Y_{1}$ the union of components on which $Z$ is non-zero and $Y_{2}$ the union of those on which it is not. From above we may assume that $Y_{1} \neq \emptyset$ else $Z_{red}=X_{red}$ and the result follows. Let $D=Z|_{Y_{1}}$ and write $L|_{Y_{1}}=L', A|_{Y_{1}}=A'$. As in the proof of previous theorem, after possibly replacing $L,D$ with a multiples, we may find $k$ such that every section of $(mkL'-kA')|_{kD}$ lifts to one of $mkL'-kA'$. 
	
	Similarly for $n>>0$ sufficiently divisible we have $\mathbf{B}((nL-kA)|_{kZ})=\BS(L|_{kZ})=\BS(L|_{Z})$ by assumption. Taking any section $s$ of $(mkL-kA)|_{kZ}$, we may restrict to a section on $kD$ and then lift to $s'$ a section of $k(mL'-A')$. By construction $s|_{Y_{2}},s'$ glue along $Y_{1}\cap Y_{2}\subseteq D$ to give a corresponding section of $k(mL-A)$ and the result follows. We may perform this gluing by \autoref{powers}.
\end{proof}


\begin{lemma}\label{red-eq}
	
	Let $X$ be a projective scheme over an excellent Noetherian base with $L$ a nef line bundle on $X$. Suppose that $\BS(L)=\mathbb{E}(L)$ and that $Z$ is closed subscheme of $X$ with $\mathbb{E}(L) \subseteq Z$. Then $\BS(L|_{Z})=\mathbb{E}(L|_{Z})$.
	
\end{lemma}

\begin{proof}
	
	Choose $m> 0$, and $A$ ample on $X$ with $\BS(L)=\BB(mL-A)$ and $\BS(L|_{Z})=\BB((mL-A)|_{Z})$. Then we have that $\BB((mL-A)|_{Z}) \subseteq \BB(mL-A)\cap Z$ by restriction.
	
	On the other hand, since $\mathbb{E}(L) \subseteq Z$, we have that $\mathbb{E}(L|_{Z}) = \mathbb{E}(L)$. Hence we have that $$\BS(L|_{Z})\subseteq \BB((mL-A)|_{Z})\subseteq \BB(mL-A)\cap Z =\mathbb{E}(L) \cap Z =\mathbb{E}(L|_{Z}).$$
	It is always the case that $\mathbb{E}(L|_{Z}) \subseteq \BS(L|_{Z})$ and hence equality holds.
	
	
\end{proof}


\begin{theorem}\label{ext}
	
	Let $X$ be a projective scheme over an excellent Noetherian base $S$ with $L$ a nef line bundle on $X$. Suppose that $\BS(L|_{X_{\mathbb{Q}}})=\mathbb{E}(L|_{X_{\mathbb{Q}}})$. Then in fact $\BS(L)=\mathbb{E}(L)=\BS(L|_{X_{red}})$.
	
\end{theorem}

\begin{proof}
	
	It is immediate that $\mathbb{E}(L)\subseteq \BS(L)$. Since $\mathbb{E}(L)=\mathbb{E}(L|_{X_{red}})$ it suffices to show only that $\BS(L) \subseteq \mathbb{E}(L)$. We may assume therefore that $\mathbb{E}(L) \neq X$ and $L$  is big, or the result follows immediately.
	
	The proof will be by Noetherian induction. So we assume that the result holds on every proper closed subscheme of $X$. The question is local on the base, so we may assume that $S$ is a $\mathbb{Z}_{(p)}$ scheme for some $p > 0$. Note that by \autoref{red-eq} we have that $\mathbb{E}(L|_{X_{red,\mathbb{Q}}})=\BS(L|_{X_{red,\mathbb{Q}}})$\\
	\\
	\textbf{Step 1: Find a non-vanishing section $t$ of $mL-A$.}\\
	
	Take $A$ ample with $\SB(mL-A)=\BS(L)$ and $\SB((mL-A)|_{X_{red}})=\BS(L|_{X_{red}})$. Suppose first that $\mathbb{E}(L)=\SB((mL-A)|_{X_{red,\mathbb{Q}}}) \neq X_{red,\mathbb{Q}}$, in which case $\SB((mL-A)|_{X_{\mathbb{Q}}}) \neq X_{\mathbb{Q}}$ since we have $\SB((mL-A)|_{X_{\mathbb{Q}}})=\mathbb{E}(L)$. Hence there is some non-zero section $t$ of $mL-A$ which does not vanish everywhere on $X_{red}$. Otherwise $\SB((mL-A)|_{X_{red,\mathbb{Q}}}) = X_{red}$, that is $$H^{0}(X_{red,\mathbb{Q}},k(mL-A)|_{X_{red,\mathbb{Q}}}) =0$$ for all $k$. 
	
	Since $\mathbb{E}(L|_{X_{red}})=\mathbb{E}(L) \neq X$, $L|_{X_{red}}$ is still big. Now by \autoref{bigsecs} there is a section $s\in H^{0}(X_{red},(mL-A)|_{X_{red}})$ which does not vanish on any component on which $L|_{X_{red}}$ is big. In particular it does not vanish everywhere. Then since $H^{0}(X_{red,\mathbb{Q}},(mL-A)|_{X_{red,\mathbb{Q}}}) =0$ we may use \cite[Proposition 3.5]{witaszek2020keel} to lift $s$ to a section $t$ of $H^{0}(X,p^{e}(mL-A))$ for some $e > 0$ with $t|_{X_{red}}=s^{p^{e}}$, so $t$ is precisely the non-vanishing section we seek.\\
	\\
	\textbf{Step 2: Reduce to $Z=Z(t)$.}\\
	
	By construction we have $\mathbb{E}(L)\subseteq Z$, since $\BS(L) \subseteq Z$. By \autoref{red-eq}, then, we have that $\BS(L|_{kZ_{\mathbb{Q}}})=\mathbb{E}(L|{kZ_{\mathbb{Q}}})$ for $k \geq 1$, so the hypotheses of the theorem are still satisfied by $kZ$. Hence by the induction hypotheses we may assume $\BS(L|_{kZ})=\mathbb{E}(L|_{kZ})=\BS(L|_{Z_{red}})$ for all $k \geq 1$. Therefore we can apply \autoref{reduce} to deduce the result.
\end{proof}

\begin{remark}
	It is not clear in what generality the assumptions of this theorem should hold. Certainly if $S_{\mathbb{Q}}$ is a field they hold by \cite{birkar2017augmented}. Even when $S_{\mathbb{Q}}$ is of finite type over a field however it is not known whether the condition holds. The arguments of \cite{birkar2017augmented} do not hold in this relative setting as they rely heavily on certain cohomology groups being vector spaces over a field. One possible remedy, when $S_{\mathbb{Q}}$ is of finite type over a field, is to find a suitable compactification and reduce to the case that $X_{\mathbb{Q}}$ is projective over a field.
\end{remark}


\begin{lemma}\label{Case:SA}
	Let $X$ be a projective scheme over an excellent base $S$. Suppose that $L$ is a semiample line bundle, inducing $\pi:X \to Y$ with $\pi_{*}\ox=\mathcal{O}_{Y}$. Then we have equalities \[\mathbb{E}(L)=\BS(L)=\text{Exc}(\pi)\]
	where $\text{Exc}(\pi)$ is the union of closed, integral subschemes $Z \subseteq X$ such that $Z \to \pi(Z)$ is not an isomorphism at the generic point.
\end{lemma}

\begin{proof}
	
	The morphism $\pi$ is proper and it's own Stein factorisation. So by Zariski's Main Theorem \cite[Tag 03GW]{stacks-project}, $\text{Exc}(\pi)$ is precisely the complement of the locus on which $\pi$ is finite, or equally the locus on which it has finite fibres.
	
	After replacing $L$ with a multiple we have $L=\pi^{*}A$ for some ample $A$ on $Y$.
	
	Take any hyperplane $H$ on $X$, let $\mathcal{I}=\pi_{*}\ox(-H)$ be the ideal sheaf induced on $Y$, so that we have $\pi_{*}(\ox(kL-H))=\ox[Y](kA)\otimes \mathcal{I}$. 
	
	Suppose that $x \in X \setminus \text{Exc}(\pi)$, then we may assume $H$ does not contain $x$ and so the co-support of $I$ does not contain $\pi(x)$. Choose $k>>0$ such that $\ox[Y](kA)\otimes I$ is globally generated. Hence there is a section $s \in H^{0}(Y,\ox[Y](kA)\otimes \mathcal{I})$ not vanishing at $\pi(x)$.
	
	However by adjunction we have natural isomorphisms $$H^{0}(Y,\ox[Y](kA)\otimes \mathcal{I}) \simeq H^{0}(Y,\pi_{*}(\ox(kL-H)))\simeq H^{0}(X,kL-H).$$ The corresponding section $s' \in H^{0}(X,kL-H)$ does not vanish at $x$ by construction.
	
	Hence we have inclusions $\mathbb{E}(L)\subseteq \BS(L)\subseteq \text{Exc}(\pi)$ and it remains to show that $\text{Exc}(\pi) \subseteq \mathbb{E}(L)$. More precisely it is enough to show that if $V$ is any closed, integral subscheme of $X$ such that $L|_{V}$ is big then $V \to \pi(V)$ is generically an isomorphism. 
	
	Suppose then that $L'=L|_{V}$ is big, so we have a section $s$ of $kL'-A$ for $k>>0$ and $A$ ample on $V$. Since $V$ is integral, by assumption, this induces an inclusion $\ox[V](A) \hookrightarrow \ox[V](kL')$. Now $\pi_{V}:V \to \pi(V)$ is generically an isomorphism if and only if it is generically finite, and hence if and only if it's Stein factorisation is so. Therefore we may freely replace $\pi_{V}$ with its Stein factorisation and assume that $\pi_{V}$ is induced by generating sections of $kL'$. Then the inclusion $\ox[V](A) \hookrightarrow \ox[V](kL')$ ensures that $\pi_{V}$ is generically an isomorphism, completing the proof.
\end{proof}


\begin{corollary}\label{Main_Loci}
	Let $X$ be a projective scheme over an excellent Noetherian base $S$ with $L$ a nef line bundle on $X$. 
	Suppose that one of the following holds:
	\begin{enumerate}
		\item $S_{\mathbb{Q}}$ has dimension $0$;
		\item $L|_{X_{\mathbb{Q}}}$ is semiample;
	\end{enumerate}
	Then $\BS(L)=\mathbb{E}(L)$.
\end{corollary}

\begin{proof}

	By \autoref{ext}, it is enough to know $\BS(L|_{X_{\mathbb{Q}}})=\mathbb{E}(L|_{X_{\mathbb{Q}}})$. In case $(1)$ this follows from \cite[Theorem 1.3]{birkar2017augmented}, since each connected component of $X_{\mathbb{Q}}$ is projective over a field. In case $(2)$ this is the content of \autoref{Case:SA}.
\end{proof}



\bibliography{refs}
\bibliographystyle{amsalpha}

\end{document}

